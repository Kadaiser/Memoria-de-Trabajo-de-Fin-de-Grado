\cleardoublepage

\chapter{Introducción}

\section{Motivación}
\label{ch:Capitulo1}

Si se quiere adquirir una solución de domótica para nuestro hogar, hoy día se dispone de muchas marcas que ofrecen dispositivos, pero, en general, para las mayorías de las marcas se prohíbe expresamente, realizar ingeniería inversa o pruebas de caja negra que permitan desarrollar productos que se conviertan en competencia de la marca. En definitiva, si se busca una solución libre, debes implementarla desde la base, o partiendo del trabajo de aquellas personas que ya proveen de un punto de arranque. ¿No es suficiente las soluciones aprotadas en el sector privado de la doótica? De hecho, son suficientes, y exceden incluso las necesidades. No se trata de un problema de carencia. El problema reside en las condiciones que hay que contratar dichas soluciones.La falta de libertad por parte del usuario final para explorar y modificar los productos contratados. Tampoco existe esa necesidad de desentrañar y conocer el funcionamiento de los dispositivos en todas las personas, pero, cuando has desarrollado la formación de una ingenieria informática, no es tan extraño sentir curiosidad y querer ampliar el conocimiento de un campo tan extenso y creciente como son las plataformas de \gls{iot}.

\vspace{1cm}

En los últimos años, la domótica ha sufrido un crecimiento acelerado gracias a la interconexión de dispositivos \gls{iot} gestionados con aplicaciones móviles. Se venden kits de domótica $Plug and Play$, que requieren únicamente de la instalación de una aplicación de movil gratuita y la adquisición de los productos ofertados por los fabricantes que operen en las plataformas de dichas aplicaciones. Una gran competencia entre empresas a surgido a raíz de este planteamiento, ofreciendo una gama extensa de dispositivos y electrodomésticos que pueden combinarse, en algunos casos incluso, interoperar entre los distintos ecosistemas. Esta disputa se está desarrollando en una etapa de incertidumbre, causada por la fase experimental de productos que se ofertan a los consumidores, ya que aún no existe una necesidad real de domótica en las personas, como ocurre, por ejemplo, con los smartphones o tables. Se siguen buscando estrategias de marketing para crear dicha necesidad mediante productos que aportan confort. Tomemos por ejemplo los sistemas de iluminación con múltiples configuraciones de intensidad, o el traking de actividad, los históricos de peso en una báscula, etcétera. Independientemente de las ideas presentadas, ya se están estableciendo unas pautas comunes en todos los actores del sector de la domótica. Es en estas pautas donde aparece nuestra preocupación a la hora de optar por las soluciones con más presencia del mercado (véase Amazon, Google o Xiamoi), que motivan en este proyecto la búsqueda de una suite de domótica que se aleje de sus practicas y tendencias. A continuación describiremos los puntos principales que mas nos preocupan, y donde encontramos unos problemas que han sido creados, para beneficio de las marcas, y en detrimento del usuario final que los paga.

\vspace{1cm}

En primer lugar, la tendencia es ofrecer dispositivos de distintos rangos de precio, que están diseñados para utilizarse los propios ecosistemas privativos de cada marca. Esto implica supeditarse a las imposiciones técnicas que cada fabricante decida ofrecer, incluyendo la forma en la que funcionan dichos dispositivos, sin poder modificar las especificaciones y funcionamientos de los mismos. En esencia, cajas negras, que desconocemos su funcionamiento interno, incluyendo la posibilidad de ampliar sus funciones o repararlos por cuenta propia (tal y como se recoge en las cláusulas de uso definidas en los manuales de todas las marcas). Se deja al usuario final a merced de un contrato establecido con los fabricantes, incluyendo su soporte de post-venta y servicio técnico. Por escenificarlo en un ejemplo análogo, si se adquiere un coche, no se impide que se pueda arreglar dicho coche con piezas genéricas, ni que se esté obligado a repararlo en la casa oficial de la marca. Si bien, los dispositivos ofertados por las grandes marcas estan basados en estandares de protocolos de comunicación, se aseguran de torcer la interoperatividad añadiendo capas de funcionamiento pensadas para inhabilitar la operatividad de un producto en el ecosistema de una marca competidora. Esta tendencia, por suerte, parace estar empezando a diluirse, y es mas facil encontrar ahora dispositivos que interoperan entre multiples plataformas simultanemente.

\vspace{1cm}

En segundo lugar, la aceptación de las políticas de uso y privacidad de las aplicaciones de las distintas plataformas, que obliga a un uso restringido de los productos según las pautas del vendedor, así como la cesión de la privacidad del usuario final. Es decir, que toda interacción del usuario con los dispositivos, así como otros datos personales obtenidos en procesos de registro de sus aplicaciones, incluyendo DCPs de carácter bajo, medios, y alto según que dispositivos, será monitorizados y enviados a servidores para ser procesados, sin tener claro con qué fin.

\vspace{1cm}

En definitiva, toda esta transferencia de datos, en todas las plataformas, esta planteada para ser usada en servicios en la nube, lo que implica un envió continuo de datos a servicios externos, con medidas de seguridad desconocidas, sin garantías reales de protección de datos y bajo la eterna dependencia de la operatividad de dichos servicios, que según que países, pueden no estar conformes a la legalidad del país en el que vive el usuario final. También hay que tener en cuenta que todas estas empresas se reservan el derecho unilateral de cambiar sus condiciones de uso y política de privacidad cuando crean convenientes sin consentimiento del usuario final. Y merece una mención especial que, las pruebas ejecutadas en las aplicaciones móviles de los ejemplos mostrados coinciden en una misma máxima: contra más restrictiva sea la configuración de privacidad por parte del usuario, menor será la utilidad de la aplicació.

\vspace{1cm}

Este documento no profundizara más en los problemas de privacidad de datos de los usuarios para estas aplicaciones, ya que, de por sí, se requiere un estudio minucioso a parte, que debe apoyarse debidamente con una base de conocimiento legal, que excede las competencias adquiridas en una carrera de vertiente tecnológica como la ingeniería informática. Sin embargo, aquellos alumnos de la facultad de ingeniería informática que hayan cursado el itinerario tecnológico del grado que incluye la materia de auditoria informática y/o ética y legislación, como es el caso de los autores de este documento, y que por consecuencia poseen una base conceptual sobre los terminos como DCPs, su tratamiento y la regulación de la LOPD y RGPD en España por parte de la AEPD, se puede concluir, que todas estas aplicaciones sin ser ilegales, se encuentran demasiado cerca de la alegalidad, o se mueven en términos demasiado ambiguos, como para considerarse realmente confiables y comprometidas con los datos privados de sus usuarios.

\vspace{1cm}

Y en último lugar, como motivación adicional para el desarrollo de un suite domótica independiente y libre, se encuentra el problema que bautizaremos como "Miles de kilómetros por metro", es decir, que para encender una interruptor a un metro del usuario, la operativa de la infraestructura necesaria para que la acción ocurra, implica que los datos de dicha acción, tienen que viajar miles de kilómetros desde el movil del usuario, hasta llegar a los servidores necesarios, para algo tan trivial como encender un interruptor. Esto carece de justificación, si no existe la intención de usar los datos del usuario para nutrir $Dataset$ para $machine learning$ o para su procesamiento en marketing, venta o traking a terceros, además, en caso de caída de la red de internet (Ya sea por el \gls{isp}, por un bloqueo del servicio en origen o destino, etc) no se puede operar los dispositivos en la red local del hogar, lo cual no está justificado. Es entendible que, ante desconexión de la red de internet no se pueda operar de manera remota la domótica del hogar, pero un router doméstico puede seguir ofreciendo operatividad a la red local, incluyendo la gestión domótica desde dentro de la red.

\vspace{1cm}

Por todo esto, queremos investigar y crear un prototipo de solución domótica (Suite domótica en adelante) que evite estas pautas anteriormente descritas. Sustentado por un software libre, con unas librerías accesibles de manera pública, basado en un hardware fácil de adquirir y que con una base fundamental de conocimientos de electrónica e informatica, dicha suite pueda ser manipulada y personalizada con relativa facilidad. Buscaremos una solución integral de domótica que incluya el software necesario para operar localmente en casa (y a través de la red de internet), con una APP, que no necesite de servicios externos, ni de la aceptación de políticas de uso privativas y opacas. En el capitulo 2 de este documento se analizaran las distitnas soluciones existente para las distintas capas fisicas y abstractas, y valoraremos cuales se acercan mas a los objetivos planteados mas adeltante en el documento. Por otro lado, para entender que tan complejo es crear una suite de domótica libre desde la base, y comprender las complejidades tecnicas de cada etapa, es probable que se tenga que crear un prototipo paso a paso, evitando los muy deseables atajos que distintos frameworks pueden ofrecer.

\vspace{1cm}

No se pretende crear una solución disruptiva en el mercado de la domótica, que enfrente las soluciones ya existentes ofertadas por otras marcas, ni aportar un protocolo nuevo de comunicaciones entre dispositivos \gls{iot}, sino en conocer los requisitos necesarios para la creación de una suite domótica y la capacidad necesaria de un usuario final para que pueda instalar su propio sistema de domótica personalizado. Esto responde tambien al propio interes de aprendizaje y compresión de tecnologias nuevas, un hábito inculcado a los estudiantes de ingenieria inculcado desde los primeros años de formación.

\vspace{1cm}

Las soluciones que pueden adquirirse actualmente en el mercado, tras lo años de prueba y error, han alcanzado un proceso de instalación muy sencillo para los usuarios finales. Esto, sin embargo, será difícil de abordar en este proyecto, ya que siempre será necesario que el usuario final disponga de conocimientos específicos de informática para interpretar los pasos que estarán documentados, peor es posible alcanzar un punto intermedio, que requiera de unos conocimientos moderados, pero que este lo suficientemente guiado y preconfigurado como para ser trivial. En todo caso, se intentará minimizar en la medida de lo posible todos los pasos necesarios para montar la infraestructura, incluyendo la instalación de software y la programación de scripts, para que el prototipo pueda ser exportado con facilidad y replicarse nuevamente ahorrando tiempo, y simplificando el proceso.

\vspace{1cm}

Crear una solución de domótica generica puede cubrir la vertiente de aprendizaje, pero se quiere aportar algun ejemplo de como esta generalidad puede cubrir necesidades de casos concretos. Esta suite debe proveer de toda la arquitectura necesaria para que su flexibilidad de configuración pueda cubrir casos de uso concretos no previstos. Se ha decidido probar el prototipo de la suite de domotica resultante de este proyecto para ayudar con el control de ambiente de una bodega situada en el sotano de una casa, y el sistema de iluminacion de dicha bodega.

\section{Objetivos}
\label{ch:Capitulo1.1}

Diseñar e implementar una solución integral de domótica modular y autocontenida, que permita mediante las indicaciones de este documento, replicar la instalación y configuración de dicha solución.
\begin{itemize}
  \item Instalación y configuración de un stack de servicios que permitan controlar la suite de domótica desde un servidor alojado en la raspberry Pi. Dicho servidor actuara como nodo principal

  \item Desarrollo de una aplicación movil que permitir al usuario interactuar mediante una API-REST con dicho servidor para ejecutar las acciones y configuraciones, que solicitando el número de permisos más básicos demuestre que el exceso de permisos solicitados por las marcas responde solo a interés empresariales y no funcionales.

  \item Implementar placas microcontroladoras los dispositivo inalámbrico sensorers y actuadores que interactúen con la suite de domótica.

  \item Desarrollar un sistema de adhesion de dispositivos a la suite domotica mediante un wizard en la aplicacion movil.
\end{itemize}

\section{Plan de trabajo}
\label{ch:Capitulo1.2}

El diseño de una suite de domótica, aun creándose desde cero, debería de poder aprovechar al máximo las tecnologías y desarrollos de software libres existentes, ya que este es, precisamente, el mayor potencial del desarrollo colaborativo tan característico del software libre, que incluyen una extensa comunidad que día a día mejoran el rendimiento y seguridad de cada uno de los modulos que puedan componer este proyecto. Esto implica un estudio previo de las diferentes opciones disponibles, y una selección del software/hardware que mejor se ajuste a nuestros objetivos, más detallados en la siguiente sección. Podemos separar las distintas fases de la siguiente forma:

\begin{enumerate}
  \item Investigación de sotfware: Incluye una evaluación de la disponibilidad de software que cubra las especificaciones que deseamos tener. Sera necesario verificar si, para cada idea de implementación ya existe una solución, y en caso de existir, valorar si merece la pena crear una implementación propia (por cuestiones de aprendizaje o versatilidad, adecuación), o utilizar la ya existente.

  \item Investigación de hardware: determinar el marco de opciones clasicas utilizadas al prototipar sistemas de \gls{iot}. Entender los rangos de precio de los mismos y su escalabilidad y protocolos de funcionamiento. Al no crearse hardware nuevo, se seleccionaran las opciones que mejor encajen con los objetivos del proyecto.

  \item Experimentar y prototipar: quellas tecnologias que sean reutilizadas deben poder conectarse entre sí, con armonía, y facilidad, siendo, en aquellos puntos que sea necesario, ajustar las configuraciones e incluir desarrollos propios que permitan a todas estas tecnologias operar como un único sistema. Implementar las soluciones propuestas para cada vertiente del proyecto, en un prototipo global, que cubra todos los objetivos propuestos. Iterar el diseño de cada módulo de manera paralela e independiente, evitando que las dificultades aisladas no bloqueen el desarrollo del resto de modulos.

\end{enumerate}

\section{Estructura del documento}
\label{ch:Capitulo1.3}

El documento se estructura como sigue:

\begin{itemize}
  \item El capitulo 2 evalua la actual situacion tecnologica y planteamientos de desarrollo disponibles para crear una suite de domotica libre.

  \item El capitulo 3 se centra en la definicion de propuesta para crear un prototipo de la suite de domotica.

  \item El capitulo 4 contiene el diseño de una arquitectura \gls{iot} con aplicacion movil para operar la suite, asi como el criterio de seleccion de equipamiento.

  \item El capitulo 5 se dedica al diseño del software que debe correr dentro de la arquitectura asi como su configuracion e implentación.

  \item El capitulo 6 propone casos de uso, donde el prototipo opera y da solución a los problemas planteados.

  \item El trabajo concluye con unas reflexiones sobre el trabajo hecho y unas líneas de trabajo futuro.
\end{itemize}
