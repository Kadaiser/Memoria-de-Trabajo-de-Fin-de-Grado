\cleardoublepage

\chapter{Introducción}

\section{Introducción}
\label{ch:Capitulo1}
El presente documento recoge el proceso de creación de un prototipo como solución integral de domótica programable, gestionada mediante software para plataforma móvil y Web App. La naturaleza del proyecto posee una vertiente asequible y libre. Para poder cumplir con estos objetivos, será necesario que los materiales y dispositivos utilizados para su implementación sean sencillos de adquirir, fáciles de reemplazar, además de tener un bajo coste en precio de adquisición y de tiempo necesario para su instalación, todo ello usando como base software y hardware libre.

\section{Motivación}
\label{ch:Capitulo1.1}

En los últimos años, la domótica ha sufrido un crecimiento acelerado gracias a la interconexión de dispositivos con aplicaciones móviles. So ofrecen kits de domótica Plug and Play, que requieren solamente la instalación de una aplicación de movil gratuita y la adquisición de los dispositivos que consideres necesarios para tus necesidades. Una gran competencia entre empresas a surgido a raíz de este planteamiento, ofreciendo una gama casi ilimitada de dispositivos y electrodomésticos que pueden combinarse, en algunos casos incluso, interoperar entre distintas plataformas. Esta disputa se está desarrollando en una etapa de mercado aun inestable, y sin embargo ya están estableciendo unas pautas comunes en todos los actores del sector de la domótica.

En primer lugar, la tendencia es ofrecer dispositivos de distintos rangos de precio, que están diseñados para utilizarse cada uno en sus propios ecosistemas. Esto implica supeditarse a las imposiciones técnicas que cada fabricante decida vender, incluyendo la forma en la que funcionan los dispositivos, sin poder modificar las especificaciones y funcionamientos de los mismos, esto es, en esencia, cajas negras, que desconocemos su funcionamiento interno, e incluyen la imposibilidad de modificarlos o raparlos por cuenta propia, estado sujetos a el contrato establecido con los fabricantes, lo cual incluye estar a merced de sus tiempo de servicio y respuestas de mantenimiento.

En segundo lugar, la aceptación de las políticas de uso de las aplicaciones de las distintas plataformas obliga a un uso restringido dentro de las pautas del vender y la cesión de la privacidad, en prácticamente la totalidad de las capacidades de uso de la aplicación. Esto es, toda interacción del usuario con los dispositivos y otros datos personales que, en términos generales, siempre se piden en los procesos de registros incluyendo DCPs de carácter bajo, medios, y bajo según que dispositivos de carácter alto.

En último lugar, toda esta transferencia de datos en todas las plataformas esta planteado para ser usadas en servicios en la nube, lo que implica un envió de dichos datos a servicios externos, con medidas de seguridad desconocidas, sin garantías reales de protección de datos y bajo la eterna dependencia de la operatividad de dichos servicios, que según que países pueden no estar restringidos a la legalidad del país en el que vive el usuario final. Se plantea además la paradoja de "Mil kilómetros por cada metro", esto es, que para encender una bombilla  a un metro del usuario, la operativa de la infraestructura implica que la acción tiene que viajar miles de kilómetros para algo tan sencillo como encender un interruptor, lo cual carece de sentido para nostros y además, en caso de caída de la red de internet (Ya sea por el ISP, por un bloqueo del servicio en origen, etc. ) no se puede operar los dispositivos en la red local.

Por todo esto, queremos investigar el desarrollo de un prototipo de solución domótica que evite estas pautas. Sustentado por un software libre, con unas librerías accesibles de manera pública, que se sustente en un hardware fácil de adquirir y que con una base fundamental de conocimientos de electrónica puedan ser montados y manipulados y personalizados con relativa facilidad. Buscaremos una solución integral de domótica que incluya el software necesario para operar localmente en casa (y a través de la red de internet), con una APP, que no necesite de servicios externos, ni de la aceptación de políticas de uso privativas y opacas.

No se pretende crear una solución disruptiva en el mercado de la domótica, que enfrente las soluciones ya existentes ofertadas por otras marcas, ni aportar un protocolo nuevo de comunicaciones entre dispositivos IoT, sino en dar la opción a quien, disponiendo de esta documentación pueda instalar su propio sistema de domótica personalizado. Las soluciones que pueden adquirirse actualmente en el mercado, tras lo años de prueba y error, han alcanzado un proceso de instalación muy sencillo para los usuarios finales, esto, sin embargo, será difícil de abordar en este proyecto, ya que siempre será necesario que el usuario final disponga de conocimientos específicos de informática para interpretar los pasos que estarán documentados. En todo caso, se intentará minimizar en la medida de lo posible todos los pasos necesarios para montar la infraestructura, incluyendo la instalación de software y la programación de scripts, para que el prototipo pueda ser exportado con facilidad y replicarse nuevamente ahorrando tiempo, y simplificando la tarea.

Para disponer de una base tecnológica sólida sobre la que crear esta solución, se ha optado por utilizar una plataforma de hardware amigable como es Raspberry Pi y Arduino, que disponen de una extensa comunidad de usuarios y documentación. No solo se puede reaprovechar gran parte del trabajo ya creado en programación de scripts para componentes de hardware como sensores y actuadores, sino que cumplen con el objetivo de ofrecer una base de hardware/software libre. Adicionalmente se tratará de alcanzar una cierta descentralización de los dispositivos y la propia raspberry, basándose en el concepto de nodo principal, que habitualmente se observa en las plataformas de pago.

Todo sensor/actuador que forme parte de red de dispositivos de una solución de domótica actual, es gestionada a través de este concepto de nodo. En vez de conectar los dispositivos inalámbricos a el Router de la casa, se conectan al nodo y este a su vez es quien se conecta a la red local del hogar, para asi conectarse con los servicios externos. En general, las distintas plataformas han alcanzado un acuerdo no formalizado de actuación que funciona de la siguiente forma. El usuario final compra un nuevo dispositivo, lo enciende, dejándolo en un estado de "inclusión" a la red domótica, después, desde la aplicación de movil, se indica al nodo, que se quiere añadir un nuevo dispositivo, y tras seguir las indicaciones, el dispositivo se registra en la red del nodo. Esto, sin embargo, tiene algunos inconvenientes en el proceso de "inclusión", y aunque la probabilidad es baja, puede suceder que dos nodos de distintas viviendas, que están registrando dispositivos simultáneamente, terminasen, registrando un dispositivo que no les corresponde.

Se ha planteado este problema, junto con las 3 motivaciones principales, para crear un proceso de "inclusión" de dispositivos al nodo, que parta de una conexión alámbrica (vía USB) y resuelva este inconveniente, y simplifique el proceso de las soluciones privadas, que en ocasiones pueden fallar.

Respecto a los dispositivos que se pueden incluir en la red del nodo, para el desarrollo de este proyecto nos limitaremos a un par de casos de uso, esto es un sensor de temperatura y humedad conectado directamente al nodo, incluyendo un altavoz y un sistema de luces que cubrirán un amplio espectro de opciones, y un disipativo inalámbrico de un sensor/actuador.


\section{Objetivos}
\label{ch:Capitulo1.2}

Diseñar e implementar una solución integral de domótica modular y autocontenida, que permita mediante las indicaciones de este documento, replicar la instalación y configuración de dicha solución.
\begin{itemize}
  \item Instalación y configuración de un stack de servicios que permitan controlar la suite de domótica desde un servidor alojado en la raspberry Pi.

  \item Desarrollo de una aplicación movil que permitir al usuario interactuar mediante una API-REST con dicho servidor para ejecutar las acciones y configuraciones

  \item Implementar con una placa de arduino un dispositivo inalámbrico con un sensor que interactúe con nuestra suite de domótica.

  \item Desarrollar un sistema de conexión de dispositivos inalámbricos a nuestra suite de domótica.

  \item Agrupar y exportar el proyecto en una imagen fácil de clonar en otra raspberry con un manual sencillo.
\end{itemize}

\section{Plan de trabajo}
\label{ch:Capitulo1.3}

El diseño de una suite de domótica, aun creándose desde cero, debería de poder aprovechar al máximo las tecnologías y desarrollos de software libres existentes, ya que este es, precisamente, el mayor potencial del desarrollo colaborativo tan característico del software libre, que incluyen una extensa comunidad que día a día mejoran el rendimiento y seguridad de cada uno de los modulos de los que se componen. Esto implica un estudio previo de las diferentes opciones disponibles, y una selección del software/hardware que mejor se ajuste a nuestros objetivos, más detallados en la siguiente sección. Podemos separar las distintas fases de la siguiente forma:

\begin{enumerate}
  \item Investigación: Incluye una evaluación de la disponibilidad de software que cubra las especificaciones que deseamos tener. Sera necesario verificar si, para cada idea de implementación ya existe una solución, y en caso de existir, valorar si merece la pena crear una implementación propia (por cuestiones de aprendizaje o versatilidad, adecuación), o utilizar la ya existente.

  \item Control de versiones: utilizar repositorios con control de versiones que permitan el desarrollo simultaneo de distintos elementos del proyecto sin que suponga colisiones a la hora de juntar los desarrollos, permitiendo separar y clasificar cada uno de estos elementos, para manejar distintos versionados.

  \item Experimentar: Aquellas tecnologias que sean reutilizadas deben poder conectarse entre sí, con armonía, y facilidad, siendo, en aquellos puntos que sea necesario, ajustar las configuraciones e incluir desarrollos propios que permitan a todas estas tecnologias operar como un único sistema.

  \item Prototipar: Implementar las soluciones propuestas para cada vertiente del proyecto, en un prototipo global, que cubra todos los objetivos propuestos. Iterara el diseño de cada módulo de manera paralela e independiente, evitando que las dificultades aisladas no bloqueen el desarrollo del resto de modulos.

  \item Publicar: Encapsular el prototipo en un formato exportable y fácil de instalar
\end{enumerate}

\section{Estructura del documento}
\label{ch:Capitulo1.4}

El documento se estructura como sigue:

\begin{itemize}
  \item El capitulo 2 evalua la actual situacion tecnologica y planteamientos de desarrollo disponibles para crear una suite de domotica libre.

  \item El capitulo 3 se centra en la definicion de propuesta para crear un prototipo de la suite de domotica.

  \item El capitulo 4 contiene el diseño de una arquitectura IoT con aplicacion movil para operar la suite, asi como el criterio de seleccion de equipamiento.

  \item El capitulo 5 se dedica al diseño del software que debe correr dentro de la arquitectura asi como su configuracion e implentación.

  \item El capitulo 6 propone casos de uso, donde el prototipo opera y da solución a los problemas planteados.

  \item El trabajo concluye con unas reflexiones sobre el trabajo hecho y unas líneas de trabajo futuro.
\end{itemize}
