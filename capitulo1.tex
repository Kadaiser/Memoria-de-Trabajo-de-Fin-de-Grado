\cleardoublepage

\chapter{Introducción}

\section{Introducción}
\label{ch:Capitulo1}
El presente documento recoge el proceso de creación de un prototipo como solución integral de domótica programable, gestionada mediante software para plataforma móvil y Web App. La naturaleza del proyecto posee una vertiente sostenible, asequible y libre. Para poder cumplir con estos objetivos, será necesario que los materiales y dispositivos utilizados para su implementación sean sencillos de adquirir, fáciles de reemplazar, además de tener un bajo coste en precio de adquisición y de tiempo necesario para su instalación, todo ello usando como base software y hardware libre.

\section{Motivación}
\label{ch:Capitulo1.1}

El planteamiento de instalar domótica en un edifico, requiere una planificación extensa y considerar múltiples variables que pueden complicar, en algunos casos imposibilitar, el dotar de automatismos un espacio.
El problema principal, un presupuesto elevado. En un plano secundario, todas las soluciones aportadas por el sector privado que resultan practicas no son planteadas desde una perspectiva de interoperatividad, si no como un ecosistema cerrado pensado para operar dentro de un abanico de dispositivos seleccionados por los fabricantes y software con políticas opacas.

La conectividad de dispositivos en el hogar a pasado de ordenadores/smartphones/tables conectados al router, a disponer de conectividad en electrodomésticos como la nevera, la lavadora, la televisión, o los sistemas de iluminación. Si bien esta conectividad a día de hoy es opcional para los inquilinos, sus cada vez mayores ventajas y confort aportado esta imponiéndose con fuerza en los hogares, pero siempre contratando los servicios de empresas privadas, y dentro de las opciones que quieran ofrecer.

Surge la cuestión de disponer de una solución libre, que sea públicamente accesible e implementable por quien quiera, implementándola en base a sus necesidades, que, sin disponer de un servicio técnico dedicado, sea mantenible y cuyos elementos sean fáciles de reemplazar.


----REVISAR ESTE CONTENIDO
Una vivienda con inquilinos requiere de dos suministros básicos, luz y agua. Generalmente se incluye un tercero, el gas. Para este proyecto consideraremos sólo los dos primeros, ya que resultan más sencillos de manipular y medir. Es más sencillo cerrar una llave de paso de agua o acceder al cableado eléctrico en los cajetines de las paredes que manipular las tuberías de gas, las cuales deben manipularse con extremo cuidado dada su naturaleza volátil. Esto no significa que no haya riesgos en la manipulación de los otros dos suministros, pero es más seguro incluir un dispositivo en la red eléctrica ya sea mediante enchufes o empalmes, o incluir a un grifo una bomba de agua, que manipular una cañería de gas.
--------


\section{Contexto}
\label{ch:Capitulo1.2}

Las compañías de suministros se aseguran de medir con exactitud el gasto a facturar del consumo, esto no implica que los consumidores sepan cómo y cuánto se están aprovechando dicho consumo. Existen muchos factores por los cuales se derrochan recursos en un hogar, desde malos hábitos, descuidos o una falsa sensación de control. Con un control digitalizado y una programación adecuada, una casa puede ejecutar una serie de acciones que implican un menor consumo de recursos, un mejor aprovechamiento del suministro de agua/luz y por consecuencia, un ahorro en el bolsillo del ciudadano a la par que un modo de vivir más sostenible.

El gasto medio de un hogar en España [según fuentes* INE o bien Instituto para la Diversificación y Ahorro de la Energía (IDAE)]. Existen distintas estrategias para reducir estos costes, algunas impulsadas por iniciativas gubernamentales como la introducida por el Ministerio de Industria, Turismo y Comercio con el bono para canjear de forma gratuita una bombilla convencional por dos de bajo consumo [insertar referencia Ej: \url{http://www.rtve.es/noticias/20090706/industria-comienza-repartir-gratis-20-millones-bombillas-bajo-consumo/283773.shtml}]. Otras mediante subvenciones como las cedidas por el Ministerio de Energía, Turismo y Agenda Digital a través del IDAE[ ref Ej: \url{http://www.idae.es/ayudas-y-financiacion/para-rehabilitacion-de-edificios-programa-pareer/programa-de-ayudas-para-la}].Otras basadas en la adquisición de electrodomésticos con alta eficiencia energética y consejos de `vox populi` que orientan un uso adecuado del agua y la luz. Todas estas estrategias son adecuadas y pueden combinarse entre sí para reducir el consumo y maximizar el ahorro. De hecho, las pautas habitualmente recomendadas a los consumidores desde asociaciones como la OCU [EJ: \url{https://www.ocu.org/vivienda-y-energia/gas-luz/consejos/trucos-ahorrar-energia}] o los de sentido común, son los que pueden repercutir con mayor impacto en la búsqueda de cumplir estos objetivos. (Explicar aquí como los consejos comunes se los salta la gente por el forro, no porque no quieran, sino por la falta de disciplina y como un proceso automático puede ayudar a lograr este ahorro).

Actualmente existen soluciones ofertadas por empresas privadas cuyos productos tiene por objetivo aportar confort al hogar mediante el control remoto o la programación de los dispositivos electrónicos del hogar; sin embargo, sus productos privativos y prácticas empresariales poco éticas despiertan suspicacias entre los consumidores, los cuales no quieren renunciar a su privacidad a cambio de automatismos.

Otro de los inconvenientes de soluciones privadas se encuentra en el ecosistema de dispositivos y software que cada compañía quiere imponer, no siendo éstos compatibles habitualmente entre distintas compañías, esto es, su interoperatividad, tendencia que sin embargo, parece estar cambiando.

\section{Objetivos}
\label{ch:Capitulo1.3}

Diseñar una solución tecnológica modular y personalizable a las necesidades de cada hogar en función del presupuesto y tipo de vivienda.

Definición de casos de uso. (al menos 3)
Implementación efectiva de servicios
Scripting de bajo nivel
Scripting de comunicación con la BBDD
Estrategia de APP para lado cliente

SEPTIEMBRE:
Control de versiones con GitFlow
Look and feel de la APP (Vista principal HOME, vista MORE, vista LOG)
Infraestructura operativa
Un caso de uso operativo a nivel de BBDD

\section{Plan de trabajo}
\label{ch:Capitulo1.4}

Dado que el servidor se basará en un conjunto de scripts de lenguaje de python que se encargarán de obtener información de los sensores, para mantener un control de versiones básico usaremos Git. El control de repositorio de Git se instala mediante el paquete 'git-core'. Pueden crearse un servidor de Git que almacene el control de versiones dentro de la misma raspberry sin necesidad de depender de servicios a terceros como GitHub o GitLab.

Instalamos en el nodo principal los paquetes de pyton:
\begin{verbatim}
sudo apt-get install build-essential python-dev
\end{verbatim}

Aquí hablaríamos de cómo hemos organizado el proyecto, nuestras fechas, metodología ágil, sistema de repositorios, aplicaciones necesarias para el desarrollo del trabajo, bla bla bla...

\section{Estructura del documento}
\label{ch:Capitulo1.5}
