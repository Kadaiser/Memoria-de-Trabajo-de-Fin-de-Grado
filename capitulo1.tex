\cleardoublepage

\chapter{Motivación}

%\label{ch:chapter1}
\label{makereference}
El presente documento recoge el proceso de creación del prototipo de una solución integral de domótica programable, gestionada mediante software para plataforma movil y Web App. Como valor añadido al producto, se plantea cubrir una serie de objetivos adicionales que permitirán a esta solución ahorrar costes económicos y gastos innecesarios de los recursos naturales. La naturaleza del proyecto posee una vertiente ecológica, sostenible, asequible y libre. Para poder cumplir con estos valores, será necesario que los materiales utilizados para su implementación sean amigables con el medio ambiente, además de encontrar un equilibrio razonable en los costes de implementación y el uso como base fundamental de software/hardware libre.

\section{Motivación}
\label{makereference1.1}

Generalmente el consumo de recursos naturales así como el cuidado del medio ambiente se presenta como una responsabilidad de los gobiernos y las empresas, pero los propios ciudadanos somos parte del problema también. Aunque las compañías se aseguran de medir con exactitud el gasto a facturar del suministro consumido, esto no implica que los consumidores sepan como y cuanto están aprovechando dicho consumo. Existen muchos factores por los cuales se derrochan recursos en un hogar, desde malos hábitos, descuidos o una falsa sensación de control.

Con un control digitalizado y una programación adecuada, una casa puede ejecutar una serie de acciones que implican un menor consumo de recursos, un mejor aprovechamiento del suministro de agua/luz y por consecuencia, un ahorro en el bolsillo del ciudadano a la par que un modo de vivir más sostenible.

Actualmente existen soluciones ofertadas por empresas privadas cuyos productos tiene por objetivo aportar confort al hogar mediante el control remoto o la programación de los dispositivos electrónicos del hogar, sin embargo, sus productos privativos y prácticas empresariales poco éticas despiertan suspicacias entre la población, la cual no quiere renunciar  a su privacidad a cambio de automatismos.

Los autores de este documento coinciden en la necesidad de crear soluciones integrales o parciales para el hogar, que faciliten la tarea de gestión doméstica.


\section{Objetivos}
\label{makereference1.2}

Diseñar una solución tecnológica modular y personalizable a las necesidades de cada hogar en función del presupuesto y tipo de vivienda. 

Definición de casos de uso. (al menos 3)
Implementación efectiva de servicios
Scripting de bajo nivel
Scripting de comunicación con la BBDD	
Estrategia de APP para lado cliente

SEPTIEMBRE:
Control de versiones con GitFlow
Look and feel de la APP (Vista principal HOME, vista MORE, vista LOG)
Infraesctrucuta opertiva
Un caso de uso operativo a nivel de BBDD


\section{Equipación}
\label{makereference1.3}

Existen múltiples niveles de implementación del prototipo planteado en este documento. En función de la cantidad de módulos operativos se requerirá una mayor cantidad de hardware para la sensorización y los actuadores implicados. El nivel mas básico e imprescindible implica un ordenador con SO Linux que actuara como controlador de la infraestructura de dispositivos que dan forma al sistema en su conjunto. Esto permite un gran abanico de opciones, el presente documento explica la implementación mediante un ordenador compacto de la marca Raspberry Pi. La selección particular de este hardware se apoya en 2 características esenciales en el desarrollo de este proyecto, es barato y de bajo consumo eléctrico. Además goza de presencia en el mercado de dispositivos, por lo cual es fácil de adquirir y a acumulado una extensa comunidad de usuarios que facilitan su uso mediante tutoriales y experimentos. Como motivos adicionales se encuentra su reducido tamaño que permite ubicarlo con facilidad en lugares estrechos o difícilmente accesibles y su imperceptible ruido al operar.

\section{Criterio de selección del equipamiento}
\label{makereference1.4}
El modelo concreto para el desarrollo del prototipo es Rapsberry Pi3b+ que dispone de capacidad de procesador y memoria RAM suficiente para operar todo el software necesario, este modelo no integra un almacenamiento interno para el usuario, pero su interfaz incluye una ranura de tarjetas micro-SD compatible con todas las opciones de tamaño de almacenamiento disponibles en el mercado. Se precisan de al menos 16 GB de espacio disponible en la memoria del sistema, y es recomendable exceder este mínimo siempre que sea posible, ya que la acumulación de datos con el paso del tiempo por parte del dispositivo crecerá indefinidamente.


\section{Esquema de trabajo}
\label{makereference1.5}
Aquí hablaríamos de como hemos organizado el proyecto, nuestras fechas, metodología ágil, sistema de repositorios, aplicaciones necesarias para el desarrollo del trabajo, bla bla bla...
