\cleardoublepage

\chapter{Motivación}

%\label{ch:chapter1}
\label{makereference}
El presente documento recoge el proceso de creación del prototipo de una solución integral de domótica programable, gestionada mediante software para plataforma movil y Web App. Como valor añadido al producto, se plantea cubrir una serie de objetivos adicionales que permitirán a esta solución ahorrar costes económicos y gastos innecesarios de los recursos naturales. La naturaleza del proyecto posee una vertiente ecológica, sostenible, asequible y libre. Para poder cumplir con estos valores, será necesario que los materiales utilizados para su implementación sean amigables con el medio ambiente, además de encontrar un equilibrio razonable en los costes de implementación y el uso como base fundamental de software/hardware libre.

\section{Motivación}
\label{makereference1.1}

Generalmente el consumo de recursos naturales o el cuidado del medio ambiente se presenta como una responsabilidad de los gobiernos y las empresas, pero los propios ciudadanos somos parte del problema también. Aunque las compañías se aseguran de medir el gasto a facturar de sumisitos, esto no implica un control real y un aprovechamiento de los mismos por parte de los consumidores. Existen muchos factores por los cuales pueden derrocharse recursos en un hogar, desde malos hábitos, descuidos o falsa sensación de control.

Con un control digitalizado y una programación adecuada, una casa puede ejecutar una serie de acciones que implican un menor consumo de recursos, un mejor aprovechamiento del suministro de agua/luz y por consecuencia, un ahorro en el bolsillo del ciudadano a la par que un modo de vivir más sostenible.

Actualmente existen soluciones ofertadas por empresas privadas cuyos productos tiene por objetivo aportar confort al hogar mediante el control remoto o la programación de los dispositivos electrónicos del hogar, sin embargo, sus productos privativos y prácticas empresariales poco éticas despiertan suspicacias entre la población, la cual no quiere renunciar  a su privacidad a cambio de automatismos.


\section{Objetivos}
\label{makereference1.2}

Diseñar una solución tecnológica modular y personalizable a las necesidades de cada hogar en función del presupuesto y tipo de vivienda. 

Definición de casos de uso. (al menos 3)
Implementación efectiva de servicios
Scripting de bajo nivel
Scripting de comunicación con la BBDD	
Estrategia de APP para lado cliente

SEPTIEMBRE:
Control de versiones con GitFlow
Look and feel de la APP (Vista principal HOME, vista MORE, vista LOG)
Infraesctrucuta opertiva
Un caso de uso operativo a nivel de BBDD


\section{Making a Citation}
\label{makereference1.3}