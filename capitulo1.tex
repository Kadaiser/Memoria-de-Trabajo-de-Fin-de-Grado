\cleardoublepage

\chapter{Motivación}

%\label{ch:chapter1}
\label{makereference}
El presente documento recoge el proceso de creación del prototipo de una solución integral de domótica programable, gestionada mediante software para plataforma móvil y Web App. Como valor añadido al producto, se plantea cubrir una serie de objetivos adicionales que permitirán a esta solución ahorrar costes económicos y gastos innecesarios de los recursos naturales. La naturaleza del proyecto posee una vertiente ecológica, sostenible, asequible y libre. Para poder cumplir con estos valores, será necesario que los materiales utilizados para su implementación sean amigables con el medio ambiente, además de encontrar un equilibrio razonable en los costes de implementación y el uso como base fundamental de software/hardware libre.

\section{Motivación}
\label{makereference1.1}

Generalmente el consumo de recursos naturales así como el cuidado del medio ambiente se presenta como una responsabilidad de los gobiernos y las empresas, pero los propios ciudadanos somos parte del problema también. Aunque las compañías se aseguran de medir con exactitud el gasto a facturar del suministro consumido, esto no implica que los consumidores sepan como y cuanto están aprovechando dicho consumo. Existen muchos factores por los cuales se derrochan recursos en un hogar, desde malos hábitos, descuidos o una falsa sensación de control.

Con un control digitalizado y una programación adecuada, una casa puede ejecutar una serie de acciones que implican un menor consumo de recursos, un mejor aprovechamiento del suministro de agua/luz y por consecuencia, un ahorro en el bolsillo del ciudadano a la par que un modo de vivir más sostenible.

Actualmente existen soluciones ofertadas por empresas privadas cuyos productos tiene por objetivo aportar confort al hogar mediante el control remoto o la programación de los dispositivos electrónicos del hogar, sin embargo, sus productos privativos y prácticas empresariales poco éticas despiertan suspicacias entre la población, la cual no quiere renunciar  a su privacidad a cambio de automatismos.

Los autores de este documento coinciden en la necesidad de crear soluciones integrales o parciales para el hogar, que faciliten la tarea de gestión doméstica.


\section{Objetivos}
\label{makereference1.2}

Diseñar una solución tecnológica modular y personalizable a las necesidades de cada hogar en función del presupuesto y tipo de vivienda.

Definición de casos de uso. (al menos 3)
Implementación efectiva de servicios
Scripting de bajo nivel
Scripting de comunicación con la BBDD
Estrategia de APP para lado cliente

SEPTIEMBRE:
Control de versiones con GitFlow
Look and feel de la APP (Vista principal HOME, vista MORE, vista LOG)
Infraesctrucuta opertiva
Un caso de uso operativo a nivel de BBDD


\section{Equipación}
\label{makereference1.3}

Existen múltiples niveles de implementación del prototipo planteado en este documento. En función de la cantidad de módulos operativos se requerirá una mayor cantidad de hardware para la sensorización y los actuadores implicados. El nivel mas básico e imprescindible implica un ordenador con SO Linux que actuara como controlador de la infraestructura de dispositivos que dan forma al sistema en su conjunto. Esto permite un gran abanico de opciones, el presente documento explica la implementación mediante un ordenador compacto de la marca Raspberry Pi. La selección particular de este hardware se apoya en 2 características esenciales en el desarrollo de este proyecto, es barato y de bajo consumo eléctrico. Además goza de presencia en el mercado de dispositivos, por lo cual es fácil de adquirir y a acumulado una extensa comunidad de usuarios que facilitan su uso mediante tutoriales y experimentos. Como motivos adicionales se encuentra su reducido tamaño que permite ubicarlo con facilidad en lugares estrechos o difícilmente accesibles y su imperceptible ruido al operar.

\section{Criterio de selección del equipamiento y Software}
\label{makereference1.4}
El modelo concreto para el desarrollo del prototipo es Rapsberry Pi3b+ que dispone de capacidad de procesador y memoria RAM suficiente para operar todo el software necesario, este modelo no integra un almacenamiento interno para el usuario, pero su interfaz incluye una ranura de tarjetas micro-SD compatible con todas las opciones de tamaño de almacenamiento disponibles en el mercado. Se precisan de al menos 16 GB de espacio disponible en la memoria del sistema, y es recomendable exceder este mínimo siempre que sea posible, ya que la acumulación de datos con el paso del tiempo por parte del dispositivo crecerá indefinidamente.

Tambien sera encesario incluir que hardware y que software (SO, paquetes, entronos de desarrollo, etc) usaremos y una jutificación apra cada caso.


\section{Esquema de trabajo}
\label{makereference1.5}
Aquí hablaríamos de como hemos organizado el proyecto, nuestras fechas, metodología ágil, sistema de repositorios, aplicaciones necesarias para el desarrollo del trabajo, bla bla bla...

\section{Instalación y configuración del nodo principal}
\label{makereference1.6}
 Utilizando los repositorios de distribuciones oficiales de Sistemas operativos de Raspberry Pi, descargamos la versión "Lite" de Raspbian. Para establecer una conexión SSH por terminal es necesario crear un fichero con nombre "ssh" en la raiz de la unidad de almacenamiento donde previamente se haya montado la imagen descargada. La distribución de Raspbian originalmente estaba configurada por defecto con la conexión de SSH abierta en el puerto 22 pudiendo accederse con el usuario \path{pi} y la contraseña \path{raspberry}. Este dato era ignorado por los usuarios menos experimentados y esto supuso una brecha de seguridad en todos las distribuciones que no fueron configuradas a posteriori por los usuarios según las indicaciones de la propia \href{https://www.raspberrypi.org/documentation/configuration/security.md}{ documentación de Raspberry}. En el primer arranque del SO de la Raspberry se establecerán las configuraciones básicas para las sucesivas conexiones SSH basadas en autenticación con claves privadas.

 De las estrategias disponibles para esta configuración, se crearan las claves en el equipo remoto que se conectara a la Raspberry, entregando mediante la primera conexión SHH con terminal la clave pública y almacenando la clave privada en el equipo remoto, reduciendo así el riesgo de ser expuesta fuera del dominio local del equipo. Para disponer de flexibilidad de conexión independientemente del SO del equipo remoto, la clave privada tendra un formato OpenSSH, fácil de incluir en SO Windows ya sea mediante conversión de la clave a formato PPK o como fichero accesible para aplicaciones de desarrollo, transferencias de ficheros, y/o control de versiones que integran conexiones SSH configurables (GitHub, Filezila, Eclipse, etc).

En Windows puede utilizarse aplicaciones de gestión de claves como 'puttygen'. En la sección de parámetros de generación de las claves se define SSH-2 RSA de 2048bits y en la sección de acciones pulsamos en genérate. Tras unos movimientos aleatorios de ratón se generara la clave publica en el área de texto. Se deben guardar ambas claves mediante los botones 'save public key' y 'save private key', esta ultima sera guardada con una contraseña definida en los inputs de la aplicación para tal fin. La clave privada sera almacenada en formato PPK para ser rápidamente usada por aplicaciones de conexión por terminal remota como 'putty'. Es recomendable exportar dicho fichero a formato OpenSSH mediante la misma aplicación de generación de claves, en la sección 'Conversions' del menú desplegable y seleccionar la opción 'Export OpenSSH key', definimos  un nombre para el fichero de salida y pulsamos 'save'. Esta misma operación puede realizarse desde una terminal de un SO Linux mediante el comando 'ssh-keygen -t rsa' que generara por defecto la claves en el directorio \path{/home/username/.ssh/} bajo el nombre \path{id_rsa.pub} e \path{id_rsa} para las claves publica y privada respectivamente.

Al no disponer de interfaz mediante dispositivos I/O para una aceso local con la Rasperry, es necesario establecer una primera conexión de terminal remoto mediante SHH con usuario y cotraseña. Este primer acceso nos permite establecer las reglas de conexión que se usaran en adelante en el fichero de configuración en la ruta \path{/etc/ssh/sshd_config} asi como la configuracion de cuentas de usuarios.

Las distribuciones de Raspbian disponen del usuario por defecto 'pi'. Esta cuenta de usuario esta incluido dentro del grupo de usuarios 'sudo'. En adelante se operara con una cuenta distinta que ha de generarse manualmente y adicionalmente eliminar la cuenta del usuario 'pi' para limitar brechas de seguridad. Como primer paso, crear el usuario \path{sudo adduser kadaiser} e incluir al usuario en el grupo de usuarios 'sudo'. El fichero por defecto creado durante la instalación de la distribución situado en \path{/etc/sudoers} dispone de la directiva \path{includedir /etc/sudoers.d} que debe ser descomentada en el fichero de configuración de sudo sudoers, mediante el comando visudo. Es necesario crear un fichero en la ruta \path{/etc/sudoers.d} con el siguiente formato de nombre \path{010_kadaiser-nopasswd} cuyo contenido incuya la siguiente linea \path{kadaiser ALL=(ALL) NOPASSWD: ALL} una vez se haya habilitado la directiva. Tras realizar las comprobaciones de que el nuevo usuario puede operar sin problemas con al nueva configuración de permisos, se elimina el fichero de permisos existente en \path{/etc/sudoers} para el usuario 'pi', y su eliminación del sistema con el comando \path{sudo deluser -remove-home pi}.

Para realizar la comunicación remota por terminal en SSH de manera mas segura y comoda, incluiremos un fichero con el contenido de la clave pública en una ruta manualmente definida dentro del 'home' del usuario kadaiser.

En concreto modificaremos el puerto de entrada para redirigir la conexión del puerto por defecto 22 a un valor mas elevado (Como el 45021). Esta decisión tiene como objetivo retrasar las tecnicas de sondeo de puertos de un atacante hacia un servidor que admite conexiones externas. Un bot programado para encontrar servidores y marcarlos como objetivo de ataques escaneara puertos mediante evaluacíon de respuestas con paqueteria ICMP. Igualmente un atacante puede determinar la naturaleza de los servicios ofrecidos por un servidor mediante herramientas como 'Nmap', al establecer valores elevados en los puertos, un rastreo incremental desde los valores mas bajos llevara mas tiempo permitiendo a las soluciones de seguridad (como un WFS) del servidor detectar el ataque con margen mayor de tiempo.

En este mismo fichero establecemos unos limites concretros en los valores de tiempo de gracia \path{LoginGraceTime 5} de apenas 5 segundos, impedimos el acceso del usuario root desde una coenxion externa \path{PermitRootLogin no}, limitamos el numero de intentos de conexión \path{MaxAuthTries 3} y el numero maximo de sesiones simultneas \path{MaxSessions}. Para admitir las conexiones SSH mediante una atentificación con clave es necesario habilitar la autentificacion de clave publica \path{PubkeyAuthentication yes} y definir la ruta del fichero con la clave publica almcenada localmente en el servidor \path{AuthorizedKeysFile} path{.net/.aut} (vease que en este caso hemos definido una ruta manualmente indicando que la clave publica se encuentra en un fichero oculto nombrado 'aut' en la ruta /home/pi/.net). Como refuerzo adicional configuramos el servidor para denegar todo intento de conexión mediante contraseña plana \path{PasswordAuthentication no} \path{PermitEmptyPasswords no}, y adicionalmente limitar el acceso solo a las cuentas de usuarios designadas \path{AllowUsers kadaiser}. Definidos los nuevos cambios de configuración, es necesario reiniciar el servicio
