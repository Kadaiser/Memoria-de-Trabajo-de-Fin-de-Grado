% +--------------------------------------------------------------------+
% | Sample Chapter 3
% +--------------------------------------------------------------------+

\cleardoublepage

% +--------------------------------------------------------------------+
% | Replace "This is Chapter 3" below with the title of your chapter.
% | LaTeX will automatically number the chapters.
% +--------------------------------------------------------------------+

\chapter{This is Chapter 3}
\label{makereference3}

Here are more examples of referring to previous sections.  In
Chapter~\ref{makereference} there were several sections, including
section~\ref{makereference1.1}, section~\ref{makereference1.2},
and section~\ref{makereference1.3}.

Likewise, in Chapter~\ref{makereference2}, there are
sections~\ref{makereference2.1} and ~\ref{makereference2.2}.

Sitio de prueba para la documentación del proceso de instalación de Node y demás.

Es necesario que inicialmente se ejecute [JUSTIFICAR]
\begin{verbatim}
sudo apt-get update
sudo apt-get full-upgrade
\end{verbatim}

[JUSTIFICAR POR QUÉ NODEJS y NPM]
Instalación de la versión más reciente de node.js y npm
Descarga:
\begin{verbatim}
curl -sL https://deb.nodesource.com/setup_11.x | sudo -E bash -
\end{verbatim}
Instalación:
\begin{verbatim}
sudo apt-get install -y nodejs
\end{verbatim}
Para comprobar que la instalación se ha llevado a cabo correctamente, crearemos un proyecto de ejemplo (que descartaremos de cara a nuestro trabajo).
Crearemos la carpeta node-app-example mediante el comando mkdir y nos situaremos dentro de dicha carpeta para ejecutar el comando npm init, el cual inicializa la carpeta con la estructura necesaria para un proyecto base. Se solicitará al usuario una serie de parámetros básicos y otros opcionales, sugerimos realizar esta prueba con la siguiente información:

\begin{verbatim}
package name: (app-example) app-example
version: (1.0.0)
description: App example to check node installation was fine.
entry point: (index.js) app.js
test command:
git repository:
keywords:
author:
license: (ISC)
\end{verbatim}

Se mostrará un resumen de los parámetros elegidos tal que el siguiente y deberemos confirmar con 'yes'.

\begin{verbatim}
About to write to /home/kadaiser/node-app-example/package.json:

{
  "name": "app-example",
  "version": "1.0.0",
  "description": "App example to check node installation was fine.",
  "main": "app.js",
  "scripts": {
    "test": "echo \"Error: no test specified\" && exit 1"
  },
  "author": "",
  "license": "ISC"
}
\end{verbatim}

npm install express
npm install method-override
npm install body-parser
"cors": "^2.8.5",
"bcryptjs": "^2.4.3",
"jsonwebtoken": "^8.5.0",
"passport": "^0.4.0",
"passport-jwt": "^4.0.0",

"python-shell": "^1.0.6",

Linux:
You know what you need for you system, basically your appropriate analog of build-essential. Keep rocking!

To compile and install native addons from npm you may also need to install build tools (source):

sudo apt-get install -y build-essential
Also install libudev:

sudo apt-get install libudev-dev

"usb-detection": "^4.1.0"

Instalación de mongodb

"mongoose": "^5.3.4",

sudo apt-get install libkrb5-dev

"run-rs": "^0.4.0",


[JUSTIFICAR POR QUÉ MONGODB]
1. no es necesario actualizar tablas/columnas si quieres agregar nuevas propiedades
2. escalabilidad
3. funciona con json, luego hace la vida más fácil de cara a trabajar con node ya que evitas parsers de entrada/salida de datos.

Instalación de la versión más reciente de mongodb:
https://docs.mongodb.com/manual/tutorial/install-mongodb-on-debian/

1. Será necesario instalar primero el gestor dirmngr to perform network operations for managing and downloading certificates. Si no lo hacemos, tendremos este error:
\begin{verbatim}
  gpg: failed to start the dirmngr '/usr/bin/dirmngr': No such file or directory
\end{verbatim}

\begin{verbatim}
  sudo apt-get install dirmngr --install-recommends
\end{verbatim}

2. Import the public key used by the package management system.
\begin{verbatim}
  sudo apt-key adv --keyserver hkp://keyserver.ubuntu.com:80 --recv 9DA31620334BD75D9DCB49F368818C72E52529D4
\end{verbatim}

3. Create a /etc/apt/sources.list.d/mongodb-org-4.0.list file for MongoDB.
\begin{verbatim}
  echo "deb http://repo.mongodb.org/apt/debian stretch/mongodb-org/4.0 main" | sudo tee /etc/apt/sources.list.d/mongodb-org-4.0.list
\end{verbatim}

4. Reload local package database.
\begin{verbatim}
  sudo apt-get update
\end{verbatim}

5. Install the MongoDB packages.
\begin{verbatim}
  sudo apt-get install -y mongodb-org
\end{verbatim}

4B. This will probably fail and you will need to install the Debian version for mongo with:
\begin{verbatim}
sudo apt-get install -y mongodb
\end{verbatim}

Esto instalará el daemon mongod, el mongos, el script init asociado, y un archivo de configuración.
También instalará la shell mongo, y una serie de herramientas de MongoDB.

And to start it as a service when the raspberry pi starts just have to enter this command:
\begin{verbatim}
sudo service mongod start
\end{verbatim}

The binaries are stored in the /usr/bin/ folder while the datas are in the /var/lib/mongodb/ folder. You can check everything is ok by using the mongo shell:
\begin{verbatim}
mongo
\end{verbatim}




(COMPROBAR ESTA SECCIÓN QUE VIENE)
How to make your node project run at start of raspberry as a service:

create node file:
sudo nano /etc/init.d/customScript
with this content:

#!/bin/sh
# /etc/init.d/node

if [ true != "$INIT_D_SCRIPT_SOURCED" ] ; then
    set "$0" "$@"; INIT_D_SCRIPT_SOURCED=true . /lib/init/init-d-script
fi

### BEGIN INIT INFO
# Provides:          customScript
# Required-Start:    $all
# Required-Stop:     $all
# Default-Start:     2 3 4 5
# Default-Stop:      0 1 6
# Short-Description: Starts the DAEMON_PATH/DAEMONOPTS server
# Description:       Starts the DAEMON_PATH/DAEMONOPTS server
### END INIT INFO

export PATH=$PATH:/opt/node/bin

DAEMON_PATH="/home/pi/dev/customScript”

DAEMON=node
DAEMONOPTS="app.js"
NAME=customScript
DESC="my custom script"
PIDFILE=/var/run/$NAME.pid
SCRIPTNAME=/etc/init.d/$NAME

case "$1" in
start)
    printf "%-50s" "Starting $NAME..."
    cd $DAEMON_PATH
    PID=`$DAEMON $DAEMONOPTS > /dev/null 2>&1 & echo $!`
    #echo "Saving PID" $PID " to " $PIDFILE
    if [ -z $PID ]; then
        printf "%s\n" "Fail"
    else
        echo $PID > $PIDFILE
        printf "%s\n" "Ok"
    fi
;;
status)
    printf "%-50s" "Checking $NAME..."
    if [ -f $PIDFILE ]; then
        PID=`cat $PIDFILE`
        if [ -z "`ps axf | grep ${PID} | grep -v grep`" ]; then
            printf "%s\n" "Process dead but pidfile exists"
        else
            echo "Running"
        fi
    else
        printf "%s\n" "Service not running"
    fi
;;
stop)
    printf "%-50s" "Stopping $NAME"
    PID=`cat $PIDFILE`
    cd $DAEMON_PATH
    if [ -f $PIDFILE ]; then
        kill -HUP $PID
        printf "%s\n" "Ok"
        rm -f $PIDFILE
    else
        printf "%s\n" "pidfile not found"
    fi
;;
restart)
    $0 stop
    $0 start
;;

*)
    echo "Usage: $0 {status|start|stop|restart}"
    exit 1
esac

exit 0

damos permisos:
sudo chmod 755 /etc/init.d/
make it launch at raspberry boot:
sudo update-rc.d customScript defaults
check if services are running:
service SERVICE_NAME status

If we do not want to restart the server manually every time, then we can use node package called nodemon. It resets the server every time we change the file.
npm install nodemon --save-dev
Change the package.json file and add the following line in “scripts” object.
"scripts": {
    "start": "nodemon server.js"
},



(FACILITAR EL DESARROLLO)

Clonar repositorio del código backend para trabajar directamente sobre la RaspBerry PI.

Seguimos los pasos recomendados por Github para generarnos una nueva clave ssh, añadirla tanto al agente ssh del sistema operativo como al repositorio privado en github y utilizarla para cualquier operación git por ssh mediante https.
https://help.github.com/en/articles/generating-a-new-ssh-key-and-adding-it-to-the-ssh-agent

