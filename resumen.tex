% +--------------------------------------------------------------------+
% | Copyright Page
% +--------------------------------------------------------------------+

\newpage

\thispagestyle{empty}

\begin{center}

{\bf \Huge Resumen en castellano}

  \end{center}
\vspace{1cm}

No parece que las empresas que instalan tecnología domótica estén dispuestas a proveer de entornos aislados de la red de internet. En todo caso, la solución general pasa por procesar los datos del servicio contratado en sus servidores y solo entonces proporcionar la funcionalidad. Además, estos sistemas no son precisamente baratos, y en general se rigen por políticas de uso muy estrictas que no permiten al consumidor final experimentar las posibilidades de dichos sistemas. Se pretende entender los requisitos tecnológicos para dotar de domótica un hogar. No tiene que ser una solución integral que cubra todos los casos de uso que se puedan imaginar. Basta con determinar que tecnologías existen, que precios son realmente necesarios, y probar a crear una solución modular que resuelva algún caso particular.

\vspace{0.5cm}

Las soluciones domóticas suelen caracterizarse por ser modulares, de tal forma que pueden automatizarse procesos concretos sin necesidad de dotar a todo el hogar de domótica, pero en ocasiones, surge una necesidad concreta, que puede que la marca que el usuario ha adquirido, no sea capaz de cubrir, y se deba buscar la solución en otra marca, lo que implica un ecosistema nuevo de aplicaciones y hardware que no es necesariamente interoperable.


\vspace{1cm}

% +--------------------------------------------------------------------+
% | On the line below, repla	ce Fecha
% |
% +--------------------------------------------------------------------+

\begin{center}

{\bf \Large Palabras clave}

   \end{center}

   \vspace{0.5cm}

   SmartHome, sistema aislado, Open Software, Open Hardware, privacidad y seguridad.
