\cleardoublepage

\chapter{Propuesta}
\label{ch:Capitulo3}

Tras presentar diferentes tecnologias, software, protocolos de comunicación y hardware planteados en el ámbito de la domótica del hogar, en la búsqueda de una solución libre y de bajo coste, que pueda alcanzar los objetivos planteados para la creación de un prototipo de suite domótica, se propone:

Desarrollar un prototipo de suite domótica que opere en una red local inalámbrica cuyo router actúe como nodo principal, ejecutado en un ordenador de bajo consumo, que permita la interconexión para dispositivos inalámbricos con roles de actuadores o sensores. Además, debe ser posible operar dicha suite, desde dentro de la red local, o desde fuera de la misma mediante conexiones remotas, por una aplicación para dispositivo movil con sistema operativo Android. Si bien, la operatividad remota es opcional, se debe alcanzar la gestión de la suite en el entorno de la red local, cumpliendo asi con el objetivo de disponer de un sistema aislado y autónomo, que no dependa de servicios externos para su funcionamiento.

Para lograr dicha propuesta, será necesario considerar de entre las opciones estudiadas aquellas que se ajustan mejor al alcance de esta propuesta. Dadas las distintas capas de hardware, arquitectura de red y software que componen este proyecto, la propuesta será dividida en tres conceptos modulares, permitiendo un desarrollo individual y en paralelo respecto de cada uno.

Habiendo elegido no utilizar un framework concreto, queda a nuestra entera disposición seleccionar bajo que servicios operara nuestra solución domótica. Cualquiera de los frameworks anteriormente listados estaban sujetos a una combinación de servicios que les permitía operar dentro de sus especificaciones. En más de la mitad de ellos, se utilizaban servidores web que permitan gestionar la suite domótica vía web, o a través de una app.

Una vez alcanzada una implementación funcional del prototipo se aplicará en dos casos de usos que ejemplifican la entrada y salidas características de todo sistema de domótica, la recepción, proceso y presentación de datos de un dispositivo sensor en la aplicación movil, y la gestión de un actuador desde dicha aplicación.

\section{Módulos de la propuesta}
\label{ch:Capitulo3.2}

Se creará un nodo principal que actuara como router de la red de dispositivos de la suite domótica y servidor de la aplicación movil. Dispondrá de la capacidad de ser operado de forma remota o local, almacenara los servicios y aplicaciones necesarios para que funcione la suite domótica ejecutándose en un ordenador de bajo consumo.

Se genera distintos dispositivos sensores o actuadores que podrán incluirse en la red inalámbrica del nodo principal y podrán ser operados desde dicho nodo.

Y por último la aplicación movil (front-end) y servidor (back-end) que darán al usuario la capacidad de gestionar la suite domótica.

\section{Propuesta de casos de uso}
\label{ch:Capitulo3.3}

El primer caso de uso consistirá en una simple interacción del usuario con la suite domótica para consultar la temperatura y/o humedad de una estancia. Para ello, es necesario disponer de un dispositivo inalámbrico con un sensor que recoja las mediciones y puedan ser mostradas al usuario en su smartphone.

El segundo caso de uso cubrirá la gestión por parte del usuario de un actuador basado en un interruptor de corriente, pudiendo consultar su estado actual y alternar dicho estado, también desde un smartphone.

\section{Objetivos adicionales}
\label{ch:Capitulo3.4}

Las siguientes propuestas corresponden más a un declaración de intenciones que a objetivos necesarios para cumplir la propuesta del proyecto. Son un valor añadido y deseable siempre que no comprometan los plazos de tiempo marcados por la entrega final de este documento al director de proyecto. Se encuentran enumerados según el valor de importancia.

\begin{enumerate}

  \item Vinculación de dispositivos con la red de suite domótica mediante USB, en lugar del clásico emparejamiento WIFI.

  \item Creación de una imagen autoinstalable de para otros usuarios

  \item Integración de múltiples opciones de hardware compatibles con la suite domótica.

  \item Conexiones cifradas en la red de dispositivos y en la comunicación entre aplicación movil y servidor.

  \item Exportar la suite domótica en una imagen autoinstalable, de fácil instalación que permita replicar el prototipo creado.

\end{enumerate}
