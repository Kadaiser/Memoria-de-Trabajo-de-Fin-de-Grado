% +--------------------------------------------------------------------+
% | Appendix A Page (Optional)                                         |
% +--------------------------------------------------------------------+


\cleardoublepage

% +--------------------------------------------------------------------+
% | Enter text for your Appendix page in the space below this box.     |
% |                                                                    |
% +--------------------------------------------------------------------+

\chapter{Intrucciones de configuración}
\label{AppendiA:Key1}

\section{Proceso de configuración de conexion SSH para el Gateway}
\label{AppendiA:Key2}

En Windows puede utilizarse aplicaciones de gestión de claves como 'puttygen'. En la sección de parámetros de generación de las claves se define SSH-2 RSA de 2048bits y en la sección de acciones pulsamos en 'genérate'. Tras unos movimientos aleatorios de ratón se generará la clave pública en el área de texto. Se deben guardar ambas claves mediante los botones 'save public key' y 'save private key'. Esta última será guardada con una contraseña definida en los inputs de la aplicación para tal fin. La clave privada será almacenada en formato PPK para ser rápidamente usada por aplicaciones de conexión por terminal remota como 'PUTTY'. Es recomendable exportar dicho fichero a formato OpenSSH mediante la misma aplicación de generación de claves, en la sección 'Conversions' del menú desplegable y seleccionando la opción 'Export OpenSSH key', definimos un nombre para el fichero de salida y pulsamos 'save'. Esta misma operación puede realizarse desde una terminal de un SO Linux mediante el comando \verb|ssh-keygen -t rsa| que generará por defecto la claves en el directorio \path{/home/username/.ssh/} bajo el nombre \verb|id_rsa.pub| e \verb|id_rsa| para las claves pública y privada respectivamente.

\vspace{1cm}

Al no disponer de interfaz mediante dispositivos I/O para una acceso local con la Raspberry, es necesario establecer una primera conexión de terminal remoto mediante SSH con usuario y contraseña. Este primer acceso nos permite establecer las reglas de conexión que se usarán en adelante en el fichero de configuración en la ruta \path{/etc/ssh/sshd_config} asi como la configuracion de cuentas de usuarios.

\vspace{1cm}

Las distribuciones de Raspbian disponen del usuario por defecto \verb|pi|. Esta cuenta de usuario esta incluido dentro del grupo de usuarios \verb|sudo|. En adelante se operará con una cuenta distinta que ha de generarse manualmente y adicionalmente eliminar la cuenta del usuario \verb|pi| para limitar brechas de seguridad. Como primer paso, crear el usuario \verb|sudo adduser edomus| e incluir al usuario en el grupo de usuarios \verb|sudo|. El fichero por defecto creado durante la instalación de la distribución situado en \path{/etc/sudoers} dispone de la directiva \verb|includedir /etc/sudoers.d| que debe ser descomentada en el fichero de configuración de \verb|sudo sudoers|, mediante el comando \verb|visudo|. Es necesario crear un fichero en la ruta \path{/etc/sudoers.d} con el siguiente formato de nombre \verb|010_edomus-nopasswd| cuyo contenido incluya la siguiente linea \verb|edomus ALL=(ALL) NOPASSWD: ALL| una vez se haya habilitado la directiva. Tras realizar las comprobaciones de que el nuevo usuario puede operar sin problemas con la nueva configuración de permisos, se elimina el fichero de permisos existente en \path{/etc/sudoers} para el usuario \verb|pi|, y su eliminación del sistema con el comando \verb|sudo deluser -remove-home pi|.

\vspace{1cm}

Para realizar la comunicación remota por terminal en SSH de manera más segura y cómoda, incluiremos un fichero con el contenido de la clave pública en una ruta manualmente definida dentro del 'home' del usuario edomus.

\vspace{1cm}

En concreto modificaremos el puerto de entrada para redirigir la conexión del puerto por defecto 22 a un valor más elevado (como por ejemplo el 45021). Esta decisión tiene como objetivo retrasar las técnicas de sondeo de puertos de un atacante hacia un servidor que admite conexiones externas. Un bot programado para encontrar servidores y marcarlos como objetivo de ataques escaneara puertos mediante evaluación de respuestas con paquetería ICMP. Igualmente un atacante puede determinar la naturaleza de los servicios ofrecidos por un servidor mediante herramientas como 'Nmap', al establecer valores elevados en los puertos, un rastreo incremental desde los valores más bajos llevará mas tiempo, permitiendo a las soluciones de seguridad (como un WFS) del servidor detectar el ataque con margen mayor de tiempo.

\vspace{1cm}

En este mismo fichero establecemos unos límites concretos en los valores de tiempo de gracia \verb|LoginGraceTime 5| de apenas 5 segundos, impedimos el acceso del usuario root desde una conexión externa \verb|PermitRootLogin no|, limitamos el número de intentos de conexión \verb|MaxAuthTries 3| y el número máximo de sesiones simultáneas \verb|MaxSessions|. Para admitir las conexiones SSH mediante una autentificación con clave es necesario habilitar la autentificación de clave pública \verb|PubkeyAuthentication yes| y definir la ruta del fichero con la clave publica almacenada localmente en el servidor \verb|AuthorizedKeysFile| \path{.net/.aut} (véase que en este caso hemos definido una ruta manualmente indicando que la clave pública se encuentra en un fichero oculto nombrado \verb|aut| en la ruta \path{/home/pi/.net}). Como refuerzo adicional configuramos el servidor para denegar todo intento de conexión mediante contraseña plana \verb|PasswordAuthentication no|, y adicionalmente limitar el acceso sólo a las cuentas de usuarios designadas \verb|AllowUsers edomus|. Definidos los nuevos cambios de configuración, es necesario reiniciar el servicio.

\section{Proceso de instalación y configuración para GPIO}
\label{AppendiA:Key3}


Instalamos en el nodo principal los paquetes de pyton:
\begin{verbatim}
sudo apt-get install build-essential python-dev
\end{verbatim}

\section{Proceso de subida de sckets pacas nodeMCU desde Raspberry Pi}
\label{AppendiA:Key4}
