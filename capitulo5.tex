\cleardoublepage

\chapter{Implementación}
\label{makereference5}



\section{Instalación y configuración del nodo principal}
\label{makereference5.1}
 Utilizando los repositorios de distribuciones oficiales de Sistemas operativos de Raspberry Pi, descargamos la versión ``Lite'' de Raspbian. Para establecer una conexión SSH por terminal es necesario crear un fichero con nombre \verb|ssh| en la raiz de la unidad de almacenamiento donde previamente se haya montado la imagen descargada. La distribución de Raspbian originalmente estaba configurada por defecto con la conexión de SSH abierta en el puerto 22, pudiendo accederse con el usuario \verb|pi| y la contraseña \verb|raspberry|. Este dato era ignorado por los usuarios menos experimentados y esto supuso una brecha de seguridad en todos las distribuciones que no fueron configuradas a posteriori por los usuarios según las indicaciones de la propia \href{https://www.raspberrypi.org/documentation/configuration/security.md}{documentación de Raspberry}~\cite{securingyourraspberrypi}. En el primer arranque del SO de la Raspberry se establecerán las configuraciones básicas para las sucesivas conexiones SSH basadas en autenticación con claves privadas.

 De las estrategias disponibles para esta configuración, se crearán las claves en el equipo remoto que se conectará a la Raspberry, entregando mediante la primera conexión SSH con terminal la clave pública y almacenando la clave privada en el equipo remoto, reduciendo así el riesgo de ser expuesta fuera del dominio local del equipo. Para disponer de flexibilidad de conexión independientemente del SO del equipo remoto, la clave privada tendra un formato OpenSSH, fácil de incluir en SO Windows ya sea mediante conversión de la clave a formato PPK o como fichero accesible para aplicaciones de desarrollo, transferencias de ficheros, y/o control de versiones que integran conexiones SSH configurables (GitHub, Filezilla, Eclipse, etc). Los pasos necesarios para establecer conexiones cifradas robustas pueden encontrarse en el Anexo A seccion de implementación del Gateway.

Establecemos la capacidad de la Raspberry Pi 3 para su módulo de comunicación wifi de actuar como punto de acceso en modo NAT~\cite{raspberrypiasaccesspoint}. Se configura una red con acceso vía usuario y contraseña, con WPA2 y gestión de claves WPA-PSK. De esta forma, el nodo será capaz de desplegar una red inalámbrica que permitirá a otros dispositivos incorporarse a la suite domótica.

En orden de subir sketcs a una arduino desde una Raspeberry Pi, es necesario isntalar las paqueterias del compilador sudo apt-get install arduino-mk, tras la instalación, en la ruta /usr/sahre/arduino pueden encontrar binarios y una capeta llamada examples que permiten cargan sketchs inmediatamente para comprobar el correcto funcionamiento del la placa microcontroladora. Para compilar dicho sketcs se necesita referenciar el fichero arduino.mk. De los ejemplo podemos verificar rampidamente el correcto funcionamiento de la microcontroladores utilizamos el mas basico de los sketcs, situado en /usr/share/arduino/examples/01.Basic/Blink/Blink.ino, este ejemplo es muy basico, un loop que enciende y apaga el led integrado en la placa microcontroladora cada 1000 milisegundos. Este , es por defecto el sketch que generelamente los disitntos fabricantes de placas microcontroladoras de con procesador ATmega328P suelen dejar cargado a modo de test. Alterando el valor basico de sleep entre lineas de encendido y apagado a un valor menor como 50 milisegundos se puede comprobar si la comunicación del puerto com, y el compilador suben correctamente el codigo a la placa. Es importante verificar este punto antes de continuar y esta simple prueba cofirma que la configuración actual esta bien.

Para facilitar el proceso de subida de codigo a la placa de arduino, crearemos un Makefile hijo que enlace parametros al compilador.


Ahora bien, tengamos en cuenta que las especificaciones del modulo esp8266 de wifi conectado a arduino requieren de una alimentación de 3.3V que pueden ser suministrados por la placa microcontroladora, sin embargo, esto nos deja con un problema de intensidad en la alimentación del modulo, ya que el pin de 3.3V disponible en la placa posees un amperaje de 50mA y se requieren de unos 200mA para garatizar una comunicación estable.


mqtt
\url{https://theembeddedlab.com/tutorials/install-mosquitto-on-a-raspberry-pi}
\verb|-h BROKER -t TOPIC|

\verb|mosquittosub -h localhost -t casa/comedor/temperatura|
\verb|mosquittopub -h localhost -t casa/comedor/temperatura -m "Temperatura: 25ºC"|


usando comunicacion cifrada en nodeucm esp8266
https://github.com/esp8266/arduino-esp8266fs-plugin
cifrado tls open source
\url{https://github.com/esp8266/Arduino/blob/master/libraries/ESP8266WiFi/src/WiFiClientSecure.h#L52-L66}
