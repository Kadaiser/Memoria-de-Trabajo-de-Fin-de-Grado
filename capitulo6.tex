\cleardoublepage

\chapter{Conclusión y mejoras}
\label{ch:Capitulo6}

Se ha creado un prototipo que permite integrar soluciones de manera escalonada. Para nuestra bodega, originalmente el objetivo solo trataba de observar las temperatura y humedad de la estancia, en siguientes iteraciones, se plantea automatizar ciertos aspectos del control como la temperatura manipulando electrodomésticos que permitan controlar ambos factores. EL sistema de iluminación dispone ahora además de un control de luz mas automatizado, lo cual aporta confort, gracias al control remoto de la \gls{app}.

\vspace{1cm}

El estudio requerido para la creación de una suite domótica ha representado un reto dada la extensión de opciones tecnológicas, estándares establecidos y protocolo de arquitectura y comunicación disponibles. Aun no habiendo podido experimentar con pruebas reales todos los estudios planteados en el estado del arte, podemos confirmar una selección adecuada de tecnologías para abordar un proyecto de \gls{iot} que enfrente los problemas característicos de una suite domótica.


\section{Refuerzos de la seguridad en la comunicación inalámbrica}
\label{ch:Capitulo6.1}
La librería pubSubClient~\cite{pubsubclientapi} utilizada en la generación de \gls{sketch} esta implementada para ser simple de usar. Sin embargo, tiene limitaciones en lo que respecta al cifrado de conexiones cliente-servidor mediante el protocolo \gls{mqtt}. No posee soporte para conexiones cifradas \verb|SSL/TLS|. Existen librerías ya disponibles de este protocolo que pueden aportar estas funciones, o ampliar la utilizada en este proyecto apoyándose en otras librerías que complementan la seguridad de las conexiones. Una aproximación deseable en el proyecto consta de reforzar dos puntos concretos de las conexiones inalámbricas de la suite domótica. Primero, establecer conexiones con la red \gls{wifi} mediante librerías mas robustas como \verb|WiFiClientSecure| que admite \href{https://arduino-esp8266.readthedocs.io/en/latest/esp8266wifi/client-secure-class.html}{cifrados asimétricos} con firma y verificación. 

\vspace{1cm}

El segundo paso requiere que el \gls{broker} de información del nodo principal resuelva las conexiones mediante el protocolo \gls{tls} en lugar de \gls{tcp} simple. Esta estrategia debe estudiarse antes de su implementación, ya que este protocolo, siendo mas seguro, implica una mayor sobrecarga en la red con los \gls{handshake} de conexiones de corta duración, como ocurre con los actuadores, cuya comunicación se limita a recibir un comando y reportar un resultado. También debe considerarse la mayor carga de proceso que se aplica a los dispositivos y el propio nodo principal. En una red con una veintena de dispositivos que establecen comunicaciones de corta duración con contenidos de mensajes de apenas un par de decenas de caracteres, la sobrecarga de comunicación puede ser decrimental hasta el punto de hacer que la suite funcione anormalmente lenta. Por ello, la primera capa de cifrado en la red inalámbrica sera una ampliación obligatoria y la segunda capa debe ser estudiada antes de su aplicación.

\section{Mejora en la cobertura inalámbrica de la suite domótica}
\label{ch:Capitulo6.2}

En términos de conectividad inalámbrica, la distancia de despliegue física de dispositivos esta limitada al rango de emisión del adaptador \gls{wifi} que actúa como router. En este proyecto, el nodo principal de la suite domótica es dicho proveedor de red y los dispositivos se conectan dentro de su rango operativo. Las atenuaciones de señal son particularmente notables dentro de una casa, ya que existen múltiples elementos arquitectónicos como paredes, suelos y columnas, así como la presencia de otros dispositivos con funciones de conectividad inalámbrica. De hecho, una suite domótica, en carácter general siempre va a contar con estas dificultades como norma. Por ello debe plantearse una estrategia que establezca una zona de conexión con una señal estable y de un alcance a diez metros, a partir del cual se hace notable la caída de las señales \gls{wifi} emitidas por un punto de acceso, tal y como se recoge en este articulo~\cite{wifiatenuation} de la universidad de Stanford sobre modelado de atenuaciones de señales wifi, apartado 3, "Modeling Attenuation Indoors".

\vspace{1cm}

Aunque existen repetidores de señales que pueden ampliar el rango efectivo de una \gls{wlan}, en realidad, se están generando nuevos puntos de acceso con nombres distintos, y esto complica mucho el despliegue de nuevos dispositivos en la red de la suite domótica, teniendo que prever, que nombre de \gls{wlan} le corresponde a un dispositivo que, por no estar a un alcance óptimo del nodo principal, debe conectarse a un repetidor concreto que le suministre mejor red. Incluso si se resuelve este problema, solo estaríamos consumiendo mas tomas de corriente para alimentar dichos repetidores. Pero los chips \verb|esp8266| utilizados en este proyecto disponen de la capacidad de ser desplegados con una configuración de \gls{wifi} en malla.

\vspace{1cm}

Conocido como \gls{wifi} Mesh, una arquitectura de comunicación inalámbrica donde cada dispositivo actúa como nodo de la red, siendo cada uno un punto de acceso para otros nodos, permite desplegar dispositivos mas allá del alcance del router del nodo principal.

\section{Actualización de código OTA en dispositivos}
\label{ch:Capitulo6.3}

\section{Ampliando el nodo principal con interfaz táctil}
\label{ch:Capitulo6.4}

\section{Estableciendo una IA para la gestión de la suite domótica}
\label{ch:Capitulo6.5}