% +--------------------------------------------------------------------+
% | Dedication Page (Optional)
% +--------------------------------------------------------------------+

\newpage

\thispagestyle{empty}
\begin{center}

{\bf \Huge Prólogo}
\end{center}
\vspace{1cm}

%\pdfbookmark[0]{Prologue}{PDF_Prologue}

% +--------------------------------------------------------------------+
% | Enter the text for your dedication in the space below this box.
% +----------------
La domótica es comúnmente asociada al confort en la vivienda: persianas que suben y bajan a golpe de interruptor, luces que se encienden al pasar, equipos de climatización controlados mediante termostatos y una inagotable lista de automatismos, en dispositivos o la propia infraestructura del hogar. Esta comodidad está generalmente acompañada de restricciones técnicas y económicas para la mayoría de la gente. Así, la domótica ha terminado como uno de esos elementos distintivos que sólo la sociedad de élite se puede permitir.

\vspace{0.5cm}

La abundante proliferación de dispositivos electrónicos en el hogar en las últimas décadas ha planteado la interconexión de los mismos en sistemas centralizados con control remoto. Dicho control, en el milenio pasado, estaba reservado a una terminal en alguna pared de la vivienda, que permitía gestionar los costosos automatismos ahora pueden ser manejados desde un simple $smartphone$ para hacer, por ejemplo, unas tostadas.
Y ahora todo está interconectado, si no, pregunten a cierta empresa china cuyo “grano de arroz de un budista es tan grande como una montaña”, la cual insiste en que hasta mis zapatillas tengan WiFi. Bien, es evidente e imparable la interconexión de todos los dispositivos del planeta a la obra de ingeniería más grande de la humanidad. Es por ello que han nacido términos como el $Internet$ $de$ $las$ $Cosas$ o el $fogging$.

\vspace{0.5cm}

Por supuesto, hay motivaciones para esta aparente necesidad de conectividad de sensores y actuadores. Se ven reflejadas en el nacimiento de la $industria$ $4{.}0$ y ya hay quien habla de $humanidad$ $2{.}0$. Recolectar ingentes volúmenes de datos para luego realizar estudios conductuales, o previsiones de mercado, son algunas de las aplicaciones más demandadas en nuestra era. La domótica, en ese sentido, puede beneficiarse de estos avances para proporcionar automatismos mas allá del confort, como por ejemplo cuidar el medio ambiente y de paso, nuestro bolsillo. Y es que, en cierta forma, estamos permitiendo que el automatismo tome el control de nuestras acciones. Un algoritmo bien programado, será siempre más eficiente que el más diligente de los humanos. Un sistema de domótica adecuado no se olvidara de una luz encendida en el baño, no dejara un grifo abierto, será capaz de alarmar a los servicios de emergencia cuando no hay nadie en casa y suceda una catástrofe, aprovechara la luz de las ventanas en invierno para calentar el hogar, y cerrara las persianas en verano para refrescar.

\vspace{0.5cm}

Todo esto y mucho más dejó hace tiempo de ser una utopía. La humanidad dispone de la tecnología y el talento. Vivimos una era increíble, los coches no vuelan, pero tenemos en la palma de nuestra mano un dispositivo que es millones de veces más potente que un ordenador de los que se usaban en la NASA hace más de medio siglo. Sí, no es tan visual, pero es mucho más práctico, de hecho, acabo de pedirle a mi cafetera que me haga un té sin tocarla.
