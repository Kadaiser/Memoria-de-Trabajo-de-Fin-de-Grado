% +--------------------------------------------------------------------+
% | Dedication Page (Optional)
% +--------------------------------------------------------------------+

\newpage

\thispagestyle{empty}
\begin{center}

{\bf \Huge Prólogo}
\end{center}
\vspace{1cm}

%\pdfbookmark[0]{Prologue}{PDF_Prologue}

% +--------------------------------------------------------------------+
% | Enter the text for your dedication in the space below this box.
% +----------------
La domótica es comúnmente asociada al confort en la vivienda: persianas que suben y bajan a golpe de interruptor, luces que se encienden al pasar, equipos de climatización controlados mediante termostatos y una inagotable lista de automatismos, en dispositivos o la propia infraestructura del hogar.

\vspace{0.5cm}

La abundante proliferación de dispositivos electrónicos en el hogar, en las últimas décadas, ha planteado la interconexión de todos estos dispositivos en sistemas centralizados con control remoto. Dicho control, que en el milenio pasado, estaba reservado a una terminal en alguna pared de la vivienda, ahora pueden ser manejados desde un $smartphone$ para hacer, por ejemplo, unas tostadas con la tostadora sin tenerla a la vista. El grado de interconexión de dispositivos cada vez es más grande, empezando en hogares y empresas, y terminando en distritos y ciudades. Es por ello por lo que han nacido términos como el $Internet$ $of$ $Things$ o el $fogging$, para explicar esta red de señales invisibles que nos rodean y que dan vida artificial a los lugares en los que vivimos.

\vspace{0.5cm}

Por supuesto, hay motivaciones más que suficientes para esta aparente necesidad de conectividad de sensores y actuadores. Se ven reflejadas en el nacimiento de la $industria$ $4{.}0$ y ya hay quien habla de $humanidad$ $2{.}0$. Recolectar ingentes volúmenes de datos para luego realizar estudios conductuales, o previsiones de mercado, son algunas de las aplicaciones más demandadas en nuestra era. La domótica, en ese sentido, puede beneficiarse de estos avances para proporcionar automatismos más allá del confort, como por ejemplo cuidar el medio ambiente y de paso, nuestro bolsillo. Sin embargo, se ha impuesto entre los distribuidores de estas tecnologías, el forzar a los consumidores finales a vender su privacidad y ser fieles a su ecosistema de hardware y protocolos como condición de adquisición.

\vspace{0.5cm}

El interés por conocer el funcionamiento y requisitos reales para una solución de domótica libre es el motor principal de motivación para este proyecto. Se desea averiguar el esfuerzo necesario para crear una solución de estas características y enfrentar las capacidades obtenidas a los largo de la formación como alumnos del grado de ingeniería informática.
