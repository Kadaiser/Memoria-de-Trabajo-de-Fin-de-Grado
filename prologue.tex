% +--------------------------------------------------------------------+
% | Dedication Page (Optional)
% +--------------------------------------------------------------------+

\newpage

\thispagestyle{empty}
\begin{center}

{\bf \Huge Prólogo}
\end{center}
\vspace{1cm}

%\pdfbookmark[0]{Prologue}{PDF_Prologue}

% +--------------------------------------------------------------------+
% | Enter the text for your dedication in the space below this box.
% +----------------
La domótica, comúnmente asociada al confort en la vivienda. Persianas que suben y bajan a golpe de interruptor, luces que se encienden al pasar, equipos de climatización controlados mediante termostatos y una infinita lista de automatismos en dispositivos o la propia infraestructura del hogar. Esta comodidad generalmente está acompañada de restricciones técnicas y económicas para la mayoría de la gente, por ello, la domótica ha terminado como uno de esos elementos distintivos de la sociedad de elite que se lo puede permitir.

\vspace{0.5cm}

La abundante proliferación de dispositivos electrónicos en el hogar en las últimas décadas ha planteado la interconexión de los mismos en sistemas centralizados con control remoto. El confort que en el milenio pasado estaba reservado a una terminal en alguna pared de la vivienda, que permitía gestionar los costosos automatismos ahora pueden ser manejados desde un simple smartphone para hacer, por ejemplo, unas tostadas.

\vspace{0.5cm}

Y ahora todo está interconectado, si no, pregunten a cierta empresa china cuyo “grano de arroz de un budista es tan grande como una montaña”, la cual insiste en que hasta mis zapatillas tengan WiFi. Bien, es evidente e imparable la interconexión de todos los dispositivos del planeta a la obra de ingeniería más grande de la humanidad. Es por ello han nacido términos como el Internet de las cosas o el fogging.

\vspace{0.5cm}

Por supuesto hay motivaciones para esta aparente necesidad de conectividad de sensores y actuadores, se ven reflejadas en el nacimiento de la industria 4.0 y ya hay quien habla de humanidad 2.0. Recolectar ingentes volúmenes de datos para luego realizar estudios conductuales, o previsiones de mercado, son algunas de las aplicaciones más demandadas en nuestra era.La domótica, en este sentido puede beneficiarse de estos avances para proporcionar automatismos mas allá del confort, esto es, por ejemplo, para cuidar el medio ambiente y de paso, nuestro bolsillo. Un argumento aparentemente inconexo, pero este documento, aparte de cubrir los aspectos esperados en un TFG, demostrara que ciertamente hay mucho beneficio, al alcance de buena parte de la población mundial, con la automática en el hogar.
