\cleardoublepage

\chapter{Estado del Arte de la Domótica}
\label{ch:Capitulo2}

Como definición. el término 'domótica' proviene de la unión de las palabras $domus$ (que significa 'casa' en latín) y $-tica$ (de 'automática', palabra en griego que significa ‘que funciona por sí sola’). Se entiende por domótica al conjunto de sistemas que hacen de una vivienda un edificio inteligente, aportando servicios de gestión energética, seguridad, bienestar y comunicación, y que pueden estar integrados por medio de redes interiores y exteriores de comunicación, cableadas o inalámbricas, y cuyo control goza de cierta ubicuidad, desde dentro y fuera del hogar.


Considerar también las conclusiones de sistemas Centralizados y Descentralizados de este artículo.(\url{http://gritosdetecnologia.blogspot.com/2013/04/origen-de-la-domotica.html})

La domótica se remonta a los años 70, uno de los primeros hitos fue el \href{https://es.wikipedia.org/wiki/X10}{protocolo de comunicaciones X-10}~\cite{x10protocolwikipedia} de automatización de dispositivos en la línea eléctrica de un hogar. Con él, se puede utilizar la propia red como canal de comunicaciones mediante ráfagas de pulsos. El ancho de banda de 256 dispositivos simultáneos es una cantidad más que suficiente para interactuar con los dispositivos de un hogar, si calculamos que cada elemento susceptible de ser automatizado como por ejemplo persianas, estufas, enchufes, e interruptores ocupase un espacio de este ancho de banda, seguirían sobrando espacios en un piso de 150 metros cuadrados.

En pleno 2019 pueden adquirirse los dispositivos necesarios para instalar una red X-10 que incluye interruptores, actuadores, sensores, transmisores, interfaces y unidades de control (todos ellos necesarios para obtener el control completo de la red) por precios con rangos entre 20 a 70 euros por cada elemento. Esto supone inversión elevada si se tiene en cuenta un hogar de varias habitaciones. Hay que considerar además ciertas limitaciones como que cada controlador es capaz de manejar un número limitado de dispositivos.

En cuanto a las aplicaciones necesarias para gestionar el sistema, X-10 posee alternativas de código libre para su desarrollo como \href{http://www.minervahome.net/}{Minerva}. También existen interfaces hardware que traducen el protocolo X-10 a APIs de servicios web como el dispositivo \href{http://www.iobridge.com/}{ioBridge} que supuso una disrupción en el ámbito del IoT y la automatización del hogar.

El problema de la solución del X-10 no radica en su protocolo sino en los costes, que generalemente sobrepasan los cientos de euros para las configuraciones mas sencillas. Ademas, esta instalación requiere de un proceso de obra, ya que es necesario empalmar los componente a la red electrica del hogar. En realidad, la opción de usar dispositivos del protocolo X-10 esta comndicionada a que la red electrica del hogar se instalara durante el proceso de edificación comn vistas a utilizar este sistema. De otra forma, sera necesario planificar obras y esto complica la facilidad de crear un prototipo assequeible.

Una década después surgiría el SCE (Sistema de Cableado Estructurado) que permitía el trasporte de datos y voz, esto supuso la aparición del concepto Edificio Inteligente. Sin embargo, esto requiera de una instalación compleja difícilmente viable en edificios ya existentes, como es el caso que ocupa el alcance de este proyecto.

Con la aparición de las redes inalámbricas, y sus posteriores estándares de comunicaci como el WIFI, se popularizaron los sistemas de domótica aplicada directamente sobre los dispositivos, sin tener en cuenta la red eléctrica que los hace funcionar.


\section{Frameworks disponibles para la gestión de IoT para SmartHomes}
\label{ch:Capitulo2.1}

Un planteamiento recurrente en el diseño de una solución basada en software es acelerar el proceso de desarrollo e implementación utilizando un framework. Es una buena idea. Estas herramientas están, en su mayoría, profundamente documentadas para exprimir sus capacidades al máximo, disponen de versionados y revisiones (en mayor o menor medida) que fortalecen tanto su seguridad, robustez, resiliencia e implementación. Poseen una buena abstracción del hardware en el que se ejecutan, sus servicios son modulares, su arquitectura es escalable.  En ocasiones están basados en software libre y/o gratuito, y disponen de una comunidad activa de usuarios a los que poder preguntar dudas. Estas son cualidades muy importantes, mas alla de las capacidades técnicas que cada opción pueda ofrecer.
En el mundo del IoT es impórtate determinar el alcance en conectividad que se desea alcanzar y volumen de datos a tratar. Existen frameworks pensados para interconectar ingentes cantidades de dispositivos en grandes extensiones de terreno y bajo el peso de un abrumador volumen de datos que procesar como es el caso de las SmartCities, o la infraestructura del sector primario y secundario. En algunos casos, el alcance es tan extremo que el concepto de IoT evoluciona a IoE (Internet of Everything) y se requiere la presencia de grandes actores tecnológicos, y sus soluciones, para abarcar estos proyectos, como es el caso del framework IBM BlueMix de IBM, el Cisco Virtualized Packet Core de Cisco, AWS IoT de Amazon o Azure IoT de Miscrosoft, por mencionar algunos ejemplos de este calibre.

Evidentemente, estos frameworks no fueron diseñados pensando en reducidos entornos como los de un hogar, y aunque son compatibles, el tiempo necesario en formación para su uso queda fuera de las capacidades y expectativas de un proyecto de las características que aqui se recoge. Sin embargo, también se dispone de un amplio abanico de opciones a un alcance más acorde a lo esperado de una solución SmartHome.

Existe una serie de criterios que aplicaremos al valorar las opciones de frameworks disponibles, en consonancia con la motivación de este proyecto. Antes de realizar cualquier evaluación sobre las bondades de cada plataforma, es conveniente recordar que en la solución que esperamos crear, se intentara evitar el uso de servicios en la nube o dependencias de APIs externas. Esto responde al objetivo de aislar la suite domótica a desarrollar, de la red de internet, evitando esa dependencia para su operatividad. Por supuesto, no es el objetivo crear una plataforma desconectada, ya que, se espera poder operar de forma remota los dispositivos desde fuera del ámbito de la red local del hogar. Ademas, debera disponer de un licenciamiento de codigo libre y gratuito, en otro caso, estariamos contraviniendo la naturaleza de este proyecto.


FiWare esta catalogada como una plataforma de codigo abierto que agrupan un set de estandares universales para el contexto de gestion de datos~\cite{whatisfiware}. Se sustenta en la ejecución del framework sobre Dockers que pueden ser alojados localmente en maquinas dentro del hogar, y aunque esta solución esta orientada a procesar datos en un contexto mas extenso que una SmartHome, puede aislarse de la red de internet. Limitan la portabilidad de las aplicaciones a aquellas que se catalogen como 'Powered by FIWARE', y aunque ofrecen una interfaz estandar para los componentes que integren la solución, con el objetivo de eliminar el bloqueo del proveedor de componentes, no posee un licenciamiento de codigo libre, por ello sera descartado.

OpenHab:la opcion mas cercana a nuestro ideal (colisiones de licencia GPL con ELP)

Universaal, es una plataforma que ofrece una solcución estandarizada para desarrollar AALs (Ambient asisted living), pero la descartamos ya que en su configuración se mencionana los servicios en la nube.

pimatic: estado de beta

Calaos: Otro poyecto de startup

Domoticz:

Home Asistant : mas nube

OpenMotics: Soluciones cableadas.


MainFlux: uno libre

Sin frameworks, aunque supone un mayor esfuerzo, puede orientarse mas concreteamente a nuestros objetivos mas cercanos

\section{Stack de servicios}
\label{ch:Capitulo2.2}

MEAN vs LAMP

\section{Protocolos de comunicación y selección}
\label{ch:Capitulo2.3}


El protocolo MQTT

El protocolo AMBIENTAL


IDENTIFICAR NECESIDADES (esto es muy caro).
