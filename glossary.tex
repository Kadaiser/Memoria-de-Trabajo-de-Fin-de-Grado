\newacronym{iot}{IoT}{Internet de las Cosas}
\newacronym{ioe}{IoE}{Internet of Everything}
\newacronym{api}{API}{Application Programming Interface}
\newacronym{so}{SO}{Sistema Operativo}
\newacronym{bbdd}{BBDD}{Base de datos}
\newacronym{wifi}{WI-FI}{Wireless Fidelity}
\newacronym{usb}{USB}{Universal Serial Bus}
\newacronym{mqtt}{MQTT}{Message Queuing Telemetry Transport}
\newacronym{http}{HTTP}{HyperText Transfer Protocol}
\newacronym{https}{HTTP}{HyperText Transfer Protocol Secure}
\newacronym{io}{I/O}{Imput/Output}
\newacronym{isp}{ISP}{Proveedor de servicios de internet}
\newacronym{dcp}{DCP}{Datos de carácter personal}
\newacronym{lopd}{LOPD}{Ley orgánica de protección de datos}
\newacronym{rgpd}{RGPD}{Reglamento general de protección de datos}
\newacronym{aepd}{AEPD}{Agencia española de protección de datos}
\newacronym{paas}{PaaS}{Plataforma como servicio}
\newacronym{gb}{GB}{GigaByte}
\newacronym{cli}{CLI}{Command line interface}
\newacronym{ide}{IDE}{Integrated Development Environment}
\newacronym{json}{JSON}{JavaScript Object Notation}
\newacronym{ssh}{SSH}{Secure SHell}
\newacronym{tcp}{TCP}{Transmission Control Protocol}
\newacronym{udp}{UDP}{User Datagram Protocol}
\newacronym{tls}{TLS}{Transport Layer Security}
\newacronym{wan}{WAN}{Wide Area Network}
\newacronym{wlan}{WLAN}{wireless local area network}
\newacronym{6lowpan}{6lowpan}{IPv6 over Low power Wireless Personal Area Networks}
\newacronym{soc}{SoC}{IPv6 over Low power Wireless Personal Area Networks}
\newacronym{gpio}{GPIO}{General Purpose Input/Output}
\newacronym{qos}{QOS}{Quality of Service}


\newglossaryentry{hash}{
    name=hash,
    description={funciones que cifran una entrada}
}

\newglossaryentry{salt}{
    name=salt,
    description={bits aleatorios que se usan como una de las entradas en una función derivadora de claves}
}


\newglossaryentry{pnp}{
    name=Plug and Play,
    description={enchufar, conectar y usar}
}

\newglossaryentry{smartphone}{
    name= SmartPhone,
    description={teléfono inteligente}
}

\newglossaryentry{tablet}{
    name= Tablet,
    description={dispositivo portatil con pantalla interactiva}
}


\newglossaryentry{framework}{
    name= framework,
    description={conjunto estandarizado de conceptos, prácticas y criterios para enfocar un tipo de problemática}
}

\newglossaryentry{script}{
    name= script,
    description={archivo de procesamiento por lotes}
}

\newglossaryentry{machinelearning}{
    name= Machine Learning,
    description={método de análisis de datos que automatiza la construcción de modelos analíticos}
}

\newglossaryentry{fog}{
    name= FOG,
    description={Arquitectura que utiliza dispositivos perimetrales para realizar una gran cantidad de cómputo, almacenamiento, comunicación local y enrutado a través de la red troncal de Internet}
}

\newglossaryentry{app}{
    name= APP,
    description={Abreviatura de aplicación, generalmente orientada a aplicaciones que se ejecutan en dispositivos móviles}
}
    
\newglossaryentry{gateway}{
    name= APP,
    description={dispositivo y/o software que actúa como punto de conexión entre dispositivos inteligentes y sensores}
}

\newglossaryentry{phyton}{
    name= APP,
    description={ lenguaje de programación interpretado cuya filosofía hace hincapié en una sintaxis que favorezca un código legible}
}

\newglossaryentry{sketch}{
    name= sketch,
    description={Un sketch es el nombre que Arduino usa para un programa. Es el código que se carga y ejecuta en una placa Arduino o derivada.}
}

\newglossaryentry{broker}{
    name= broker,
    description={nodo que establece las comunicaciones de una red de suscriptores y publicadores de información para realizar notificaciones inteligentes}
}

\newglossaryentry{handshake}{
    name= handshake TLS,
    description={proceso donde dos actores comunicandose en un medio intercambian mensajes para reconocerse, verificarse, establecer los algoritmos de cifrado que usarán y acordar las claves de sesión.}
}

\newglossaryentry{wizzard}{
    name= wizzard,
    description={proceso guiado por pasos para que un usuario complete una configuración o instalación de software o dispositivos.}
}

\newglossaryentry{dataset}{
    name= Dataset,
    description={una colección de datos generalmente tabulada.}
}

\newglossaryentry{docker}{
    name= Docker,
    description={Contenedores de software para el despliegue de aplicaciones de forma automática}
}