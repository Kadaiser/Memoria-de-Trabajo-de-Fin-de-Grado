\newacronym{iot}{IoT}{Internet de las Cosas}
\newacronym{ioe}{IoE}{Internet of Everything}
\newacronym{so}{SO}{Sistema Operativo}
\newacronym{bbdd}{BBDD}{Base de datos}
\newacronym{wifi}{WI-FI}{Wireless Fidelity}
\newacronym{usb}{USB}{Universal Serial Bus}
\newacronym{mqtt}{MQTT}{Message Queuing Telemetry Transport}
\newacronym{http}{HTTP}{HyperText Transfer Protocol}
\newacronym{https}{HTTP}{HyperText Transfer Protocol Secure}
\newacronym{io}{I/O}{Imput/Output}
\newacronym{isp}{ISP}{Proveedor de servicios de internet}
\newacronym{dcp}{DCP}{Datos de carácter personal}
\newacronym{lopd}{LOPD}{Ley orgánica de protección de datos}
\newacronym{rgpd}{RGPD}{Reglamento general de protección de datos}
\newacronym{aepd}{AEPD}{Agencia española de protección de datos}

\newglossaryentry{pnp}{
    name=Plug and Play,
    description={enchufar, conectar y usar}
}

\newglossaryentry{smartphone}{
    name= SmartPhone,
    description={teléfono inteligente}
}

\newglossaryentry{tablet}{
    name= Tablet,
    description={dispositivo portatil con pantalla interactiva}
}

\newglossaryentry{framework}{
    name= framework,
    description={conjunto estandarizado de conceptos, prácticas y criterios para enfocar un tipo de problemática}
}

\newglossaryentry{script}{
    name= script,
    description={archivo de procesamiento por lotes}
}

\newglossaryentry{machinelearning}{
    name= Machine Learning,
    description={método de análisis de datos que automatiza la construcción de modelos analíticos}
}

\newglossaryentry{fog}{
    name= FOG,
    description={Arquitectura que utiliza dispositivos perimetrales para realizar una gran cantidad de cómputo, almacenamiento, comunicación local y enrutado a través de la red troncal de Internet}
    }

\newglossaryentry{app}{
    name= APP,
    description={Abreviatura de aplicación, generalmente orientada a aplicaciones que se ejecutan en dispositivos móviles}
    }